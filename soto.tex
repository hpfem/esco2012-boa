\title{An Efficient Coupled Fluid/Structure Finite Element Scheme for Blast and Impact Loads over Reinforce Concrete Structures}
\tocauthor{O. Soto} \author{} \institute{}
\maketitle
\begin{center}
{\large Orlando Soto}\\
Science Applications International Corporation (SAIC)\\
{\tt orlando.a.soto@saic.com}

\end{center}

\section*{Abstract}

An efficient finite element (FE) scheme to model coupled fluid-solid
blast and impact problems is presented. The main ingredients are: 
an improved OSS-Q1/P0 solid element (Orthogonal Sub-grid Scale
Stabilized element tri-linear in velocities and constant
piecewise-discontinuous pressures), a large-strain finite element (FE)
convective formulation, a phenomenological concrete model to deal with damage
and fracture of concrete structures (K\&C model \cite{kcpaper}),
and a general contact algorithm which
uses bin technology to perform the node-face searching operations in an
efficient manner. All the schemes, contact included, have been fully
parallelized 
and coupled using an embedded procedure with the CFD
(computational fluid dynamics) code FEFLO. Several 3D coupled CFD/CSD cases
with experimental comparisons will be presented to validate the scheme.

The main ingredient of the formulation, the OSS scheme,
 allows to obtain locking free displacement/velocity fields and,
furthermore, a totally
stable (free of spurious oscillation) stress field in each time step. This last
feature is essential to produce smooth and stable damage distributions and,
therefore, clearly defined localization bands and fracture patterns. The OSS
scheme consists in adding to the standard FE Galerkin terms an additional
contribution to improve the stability of the discrete approximation. The OSS
variational form can be written as:
\begin{eqnarray}
{\rm Galerkin}
+\int_{\Omega_t} \tau\nabla q\cdot\bigl(\nabla p
-\bm \xi)~{\rm d}\Omega=0~~~;~~~\int_{\Omega_t} {\bm v}\cdot\bm \xi~{\rm d}\Omega=
\int_{\Omega_t} {\bm v}\cdot\nabla p~{\rm d}\Omega&&
\end{eqnarray}
\noindent where ${\rm Galerkin}$ refers to the standard Galerkin terms, 
$\tau$ is the stabilization parameter, $p$ is the
pressure, and $\bm \xi$ is the projection of the pressure gradient into the FE
space. The OSS method is clearly consistent at a continuous level in the sense that the
exact solution of the original dynamic equilibrium 
problem fulfills the variational form (1). It will be shown in the final paper
that the stabilization parameter $\tau$ can be designed such that the
convergence order of the additional terms (non-Galerkin terms) is at least
fourth. Hence, the original accuracy of the standard FE approximation is not
deteriorated (the standard FE scheme is second order for linear elements). Several authors have been using this
``pressure filtering'' terms for incompressible problems, but, to our
knowledge,  this is the
first time that the method is presented for explicit blast and impact coupled
fluid/solid problems. Finally, the other ingredients of the formulation may be consulted in \cite{soto2010}.


\bibliographystyle{plain}
\begin{thebibliography}{10}

\bibitem{kcpaper}
{\sc L. Malvar and J. Crawford and J. Wesevich and D. Simons}. {A plasticity concrete material for DYNA3D}.  Int. J. Impact Engng. 19, 847--873, 1997.



\bibitem{soto2010}
{\sc O. Soto and J. Baum and R. L\"ohner}. {An efficient fluid-solid  coupled finite element scheme for weapon fragmentation simulations}. Engineering Fracture Mechanics. 77, 549-564, 2010.

\end{thebibliography}
