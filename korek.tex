\title{Electronic Structure of the Nanodiatomic Compounds CdS}
\tocauthor{M. Korek} \author{} \institute{}
\maketitle
\begin{center}
{\large Mahmoud Korek}\\
Faculty of Science, Beirut Arab University, P.O.Box 115020 Riad El Solh, Beirut 1107 2809, Lebanon\\
{\tt fkorek@yahoo.com}
\\ \vspace{4mm}{\large Khalil Badreddine}\\
Faculty of Science, Beirut Arab University, P.O.Box 115020 Riad El Solh, Beirut 1107 2809, Lebanon\\
{\tt fkorek@yahoo.com}
\\ \vspace{4mm}{\large \underline{Fatme Jardali}}\\
Faculty of Science, Beirut Arab University, P.O.Box 115020 Riad El Solh, Beirut 1107 2809, Lebanon\\
{\tt fatimajardali@yahoo.com}

\end{center}

\section*{Abstract}

The characteristics of Nanoparticles have attracted intense research lately due to to their unique chemical and physical properties. Additionally, when incorporated, they demonstrate a vast potential for practical application of the composite system. The materials have interesting applications in many fields such as electronics, optical, electro-optical devices and photo-catalytic reactions. The Semiconductor CdS Nanoparticles have been widely studied and synthesized because they have  distinctive properties and are interesting for photo reactivity and photo catalyst applications[1-2]. The properties of CdS Nanoparticles are driven mainly by two factors which are the increase in the surface to volume ratio and the drastic changes in the electronic structure of the material with decreasing particles size. Because of the lack in the study of the excited electronic states in literature, the potential energy curves, the harmonic frequency e, the inter-nuclear distance re, the rotational constant Be and the electronic energy with respect to the ground state Te have been calculated. The comparison of these values to the theoretical and experimental results for the considered electronic states available in literature shows a very good agreement.

\bibliographystyle{plain}
\begin{thebibliography}{10}

\bibitem{Matxain-Mercero-Fowler-Uglade}
{\sc Matxain JM and Mercero JM and Fowler JE and Uglade JM}. {Clusters of II -VI Materials}. J. Phys. Chem. A 107 (2004) 10502-10508 DOI:10.1021/jp037195s.



\bibitem{Chambaud-Guitou- Hayashi}
{\sc Chambaud G and Guitou M and Hayashi S}. {Specific Electronic Properties of Metallic Molecules }. Chem. Phys. (2008) 352,147-156.

\end{thebibliography}
