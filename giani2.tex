\title{High-Order/$hp$-Adaptive Discontinuous Galerkin Finite Element Methods for Compressible Fluid Flows}
\author{} \tocauthor{S. Giani} \institute{}
\maketitle
\begin{center}
{\large Stefano Giani}\\
University of Nottingham\\
{\tt stefano.giani@nottingham.ac.uk}
\\ \vspace{4mm}{\large Paul Houston}\\
University of Nottingham\\
{\tt Paul.Houston@nottingham.ac.uk}

\end{center}

\section*{Abstract}

We present an overview of some recent developments concerning the a posteriori error analysis
of h- and hp-version finite element approximations to compressible fluid flows for a specific
discretisation scheme: the hp-version of the discontinuous Galerkin finite element method.
This method is capable of exploiting both local polynomial-degree-variation (p-refinement)
and local mesh subdivision (h-refinement), thereby offering greater flexibility and efficiency
than numerical techniques which only incorporate h-refinement or p-refinement in isolation.

We shall be particularly concerned with the derivation of a posteriori bounds on the error in
certain output functionals of the solution of practical interest; relevant examples include the lift
and drag coefficients for a body immersed into a fluid, the local mean value of the field or its
flux through the outflow boundary of the computational domain, and the pointwise evaluation
of a component of the solution.

By employing a duality argument we derive so-called weighted or Type I a posteriori estimates
which bound the error between the true value of the prescribed functional, and the actual computed value. In these error estimates, the element residuals of the computed numerical solution
are multiplied by local weights involving the solution of a certain dual or adjoint problem. On
the basis of the resulting a posteriori error bound, we design and implement an adaptive finite
element algorithm to ensure reliable and efficient control of the error in the computed functional with respect to a user-defined tolerance. Our adaptive finite element algorithm decides
automatically between isotropic and anisotropic refinement either in h-refine or in p-refine.
The performance of the resulting hp-refinement algorithm is demonstrated through a series of
numerical experiments.

This research has been funded by the EU under the ADIGMA project.

\bibliographystyle{plain}
\begin{thebibliography}{10}

\bibitem{giani-paul-1}
{\sc S. Giani and P. Houston}. {Anisotropic hp-adaptive discontinuous Galerkin finite element methods for compressible fluid flows}. International Journal of Numerical Analysis and Modeling, Accepted.

\end{thebibliography}
