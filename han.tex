\title{Radiative Slab Heating Analysis for Various Fuel Gas Compositions in an Walking Beam Type Reheating Furnace}
\tocauthor{S. H. Han} \author{} \institute{}
\maketitle
\begin{center}
{\large Sang Heon Han}\\
Division of Ocean System Engineering, KAIST\\
{\tt Korea}

\end{center}

\section*{Abstract}

A transient radiative slab heating analysis was performed to investigate the effect of various fuel mixtures on the performance of an axial-fired reheating furnace. The various fuel mixtures tested were assumed to be attained by mixing COG (Coke Oven Gas) and BFG (Blast Furnace Gas), which are the two main byproduct gases found in the integrated steel mill industry. The numerical prediction of radiative heat transfer was calculated using an FVM radiation solving method, which is a well-known and efficient method for curvilinear coordinates. The WSGGM  (Weighted Sum of Gray Gas Model) was also adopted to calculate the radiative heat transfer in composition dependent media. The entire furnace was divided into fourteen sub-zones to calculate the radiative thermal characteristics of the furnace without flow field calculations. Each sub-zone was assumed to have homogeneous media and wall temperatures. All of the media and wall temperatures were computed by calculating the overall heat balance using some relevant assumptions. The overall heat balance was satisfied when the net heat input equaled the three sources of heat loss in each sub-zone, wall loss, skid loss, and slab heating loss

\bibliographystyle{plain}
\begin{thebibliography}{10}

\bibitem{1xx}
{\sc S. H. Han and D. J. Chang and C. Huh}. {Efficiency analysis of radiative slab heating in a walking-beam-type reheating furnace}. Energy 36 (2011) 1265-1272.

\end{thebibliography}
