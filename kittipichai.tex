

\title{The Mass Minimization of Hospital Bed Structure for  Independently Seperating Left and or Right Leg }
\tocauthor{R. Kittipichai} \author{} \institute{}
\maketitle
\begin{center}
{\large Atthaphon Ariyarit}\\
Faculty of Engineering, Mahidol University, Nakhonpathom, 73110, Thailand\\
{\tt P46007035@mail.ncku.edu.tw}
\\ \vspace{4mm}{\large \underline{Rung Kittipichai}}\\
Faculty of Engineering, Mahidol University, Nakhonpathom, 73110, Thailand\\
{\tt egrkt@mahidol.ac.th}

\end{center}

\section*{Abstract}

At present, there are many types of the hospital  bed with various functions such as lifting the head section or leg section of the bed. Unfortunately, the hospital bed cannot lift either left or right leg. That means when the patient's left leg or right leg is broken, the hospital bed need to lift both legs. It cannot lift independently.  This work focuses on the hospital bed design for independently separating the left and or right leg for patient's leg splint. The paper proposes an idea to search the minimum mass of hospital bed. The design problem is a combination of sizing and shape optimization of bed structure. The aim is to minimize the mass of bed structure subject to nodal displacement, stress and buckling constraints using Genetic Algorithms (GAs) as a stochastic search method. The bed structure was modeled and analyzed by using Finite Element (FE) method. FE with 2-D beam element was selected and applied to the 3-D bed structure. The GAs and FE code using MATLAB program were developed to analyze the structure. The bed structure was modeled by using 54 beam elements with 38 nodes. The mass of bed structure was minimized by reducing the cross-section area of each element and changing some node position under the down force of 4,600 N and the left and right side force of 2,750 N. The structure was made of steel alloy with the safety factor of 2. The total design variables of the optimization problem were 73 variables in the sizing part and 4 variables in the shape part.  In GA procedure, the number of strings in a population and the number of generation was set to 200 and 500, respectively. The results showed the success in searching the minimum mass whilst the stress, nodal displacement and buckling constraints were accepted. The optimum mass for the bed structure was 28.19 kg with the maximum constraint of stress in the element. Therefore, this paper demonstrates that it is possible to design the hospital bed for independently separating left and or right leg by using GAs combined with FE analysis to search the minimum mass.

\bibliographystyle{plain}
\begin{thebibliography}{10}

\bibitem{Rung}
{\sc R. KITTIPICHAI and A. ARIYARIT}. {The sizing optimization of hospital bed structure for independently supporting left and or right leg using genetic algorithms}. International Journal of Modeling and Optimization Vol. 1, No. 2 (2011), pp. 122-128.



\bibitem{COOK}
{\sc R. D. COOK and ET AL.}. {Concepts and applications of finite element analysis}. New York: John Wiley Son (1989).



\bibitem{RAO}
{\sc S. S. RAO}. {The finite element method in engineering, 4th ed.}. Amsterdam: Boston, MA: Elsevier/Butterworth Heinemann (2005).



\bibitem{GOLDBERG}
{\sc D. E. GOLDBERG}. {Genetic algorithms in search, optimization, and machine learning}. Reading: Addison-Wesley (1989).

\end{thebibliography}

