\title{Evolutionary Algorithms and Seismic Rays Tracing}
\author{} \tocauthor{R. Soto} \institute{}
\maketitle
\begin{center}
{\large \underline{Roberto Soto}}\\
Universidad Autonoma de Nuevo Leon\\
{\tt robsotov@gmail.com}
\\ \vspace{4mm}{\large Javier Almaguer}\\
Universidad Autonoma de Nuevo Leon\\
{\tt almagerjavier@gmail.com}
\\ \vspace{4mm}{\large Javier Morales}\\
Universidad Autonoma de Nuevo Leon\\
{\tt tequilaydiamante@yahoo.com.mx}
\\ \vspace{4mm}{\large Francisco Benavides}\\
Universidad Autonoma de Nuevo Leon\\
{\tt fgbenavid@gmail.com}

\end{center}

\section*{Abstract}

The world complexity in which we are dealing with, shows problems with solutions that hardly can be solved by \emph{ab initio} models using analytic techniques. This has pushed to find alternative meta-heuristic solutions in the Evolutionary Computing field.
The seismic waves propagation context counts with a broad spectrum of tools and programing for the seismic rays tracing in stratified mediums with pre established speeds.
Try new approaches to board this sort of problems allows moving on in the design and implementation of smartest search algorithms. 
Above represents the main motivation in this work.  We present a new solution to the problem of seismic ray tracing through stratified medium. 
We used a meta-heuristics which calculates the path followed by a ray crossing several interfaces in a minimum time, according to Fermat principle, between the trigger point and the location of the sensor. 
It is suppose that the layers are isotropic with an homogeneous composition, this allows validate the Snell's law, as the equation used brings us not only the ray travel time through the layers, but also shows the coordinates where the rays are refracted from one interface to another or are reflected to the geophone direction. 
This kind of techniques, used in exploration and research of hydrocarbons, permit visualize the response of the stratified medium showing the ray's trajectory and the involved parameters can be adjustment, allowing validate the geological models and consequently carry out a better design of data acquisition. 

\bibliographystyle{plain}
\begin{thebibliography}{10}

\bibitem{Havar}
{\sc Havar Gjoystdal and et al.}. {Improved applicability of ray tracing in seismic acquisition, imaging, and interpretation}. Geophysics, Vol. 72, No 5, 2007 .



\bibitem{Cerveny}
{\sc V. Cerveny}. {Seismic ray theory}. Cambridge, University press., UK, 2001.

\end{thebibliography}
