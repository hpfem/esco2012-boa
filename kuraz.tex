\title{Domain Decomposition Method for the Nonstationary Richards' Equation Problem}
\tocauthor{M. Kuraz} \author{} \institute{}
\maketitle
\begin{center}
{\large \underline{Michal Kuraz}}\\
Czech University of Life Sciences Prague, Faculty of Environmental Sciences, Department of Water Resources and Environmental Modeling, Czech Republic\\
{\tt michal.kuraz@fsv.cvut.cz}
\\ \vspace{4mm}{\large Petr Mayer}\\
Czech Technical University in Prague, Faculty of Civil Engineering, Department of Mathematics, Czech Republic\\
{\tt pmayer@mat.fsv.cvut.cz}

\end{center}

\section*{Abstract}

The problem of predicting fluid movement in an unsaturated/saturated zone is important in many fields, ranging from agriculture, via hydrology to technical applications of dangerous waste disposal in deep rock formations.

  Richards' equation model on engineering problems typically involves solving systems of linear equations of  huge dimensions, and thus multi-thread methods are often preferred in order to decrease the computational time required. 
  
  In case of non-homogenous materials, if splitting the computational domain efficiently, the problem conditionality can be significantly improved, as  each subdomain can contain only a certain material set within some defined parameter range. For linear problems, as e.g. heat conduction, the domain splitting in such a way can be performed very easily.
  
  The problem arises in case of the non-linear Richards' equation, where the coefficients of a single material can vary within several orders of magnitude, see e.g. \cite{kuraz}. If Euler method is considered for the time integration, then a robust algorithm will be obtained if the domain is splitted adaptively over the time integration levels.
  The domain decomposition technique  considered here is the standard multiplicative Schwarz method with coarse level, see e.g. \cite{widlund}. Method of the domain decomposition adaptivity will be studied in this presentation.

\bibliographystyle{plain}
\begin{thebibliography}{10}

\bibitem{kuraz}
{\sc M. Kuraz and P. Mayer and V. Havlicek and P. Pech and J. Pavlasek}. {Dual permeability variably saturated flow and contaminant transport modeling of a nuclear waste repository with capillary barrier protection}. Applied Mathematics and Computation, In Press.



\bibitem{widlund}
{\sc A. Toselli and O. Widlund}. {Domain Decomposition - Methods Algorithms and Theory}. Springer-Verlag, Berlin, Heidelberg 2005.

\end{thebibliography}
