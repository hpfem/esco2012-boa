\title{Trefftz-Discontinuous Galerkin Methods for Time-Harmonic Maxwell's Equations}
\tocauthor{A. Moiola} \author{} \institute{}
\maketitle
\begin{center}
{\large \underline{Andrea Moiola}}\\
ETH Zurich - Seminar for Applied Mathematics\\
{\tt moiola@sam.math.ethz.ch}
\\ \vspace{4mm}{\large Ralf Hiptmair}\\
ETH Zurich - Seminar for Applied Mathematics\\
{\tt hiptmair@sam.math.ethz.ch}
\\ \vspace{4mm}{\large Ilaria Perugia}\\
Dipartimento di Matematica, Universita` di Pavia\\
{\tt ilaria.perugia@unipv.it}

\end{center}

\section*{Abstract}

The propagation and the interaction of time-harmonic electric waves are described by the Maxwell equations. The need of resolving oscillating solutions and the so-called pollution effect make their numerical discretization through standard FEM extremely expensive for high
frequencies. In Trefftz methods, the trial and the test functions are solution of the PDE inside each element, thus they are oscillating functions
and we can expect to be able to approximate the solution with smaller discrete spaces. The basis functions are often chosen as vector plane waves.

We formulate a class of Trefftz-discontinuous Galerkin (TDG) methods that includes the well-known Ultra Weak Variational Formulation
(UWVF), we study the convergence of their spectral version, and we show well-posedness and quasi-optimality in a mesh skeleton norm. A novel vector Rellich-type identity allows to prove new (wavenumber-independent) stability estimates for the considered boundary value problem in star-shaped polyhedral domains; we use them in a duality argument to prove error bounds in a mesh-independent norm.

The abstract convergence analysis is carried out for any Trefftz trial space. In the case of vector plane or spherical waves, concrete error
bounds are obtained by proving the corresponding best approximation estimates.

\bibliographystyle{plain}
\begin{thebibliography}{10}

\bibitem{MaxwellPDE}
{\sc R. Hiptmair and A. Moiola and I. Perugia}. {Error analysis of Trefftz-discontinuous Galerkin methods for the time-harmonic Maxwell equations}. To appear in Math. Comput..



\bibitem{MaxwellTrefftz}
{\sc R. Hiptmair and A. Moiola and I. Perugia}. {Stability results for the time-harmonic Maxwell equations with impedance boundary conditions}. Math. Models Methods Appl. Sci. (2011).



\bibitem{AndreaPhD}
{\sc A. Moiola}. {Trefftz-discontinuous Galerkin methods for time-harmonic wave problems}. Ph.D. thesis, ETH Zurich (2011).

\end{thebibliography}
