\title{Aero-Acoustic Simulation on Massively Parallel Octree Meshes}
\tocauthor{H. Klimach} \author{} \institute{}
\maketitle
\begin{center}
{\large \underline{Harald Klimach}}\\
German Research School for Simulation Sciences GmbH and RWTH Aachen University\\
{\tt h.klimach@grs-sim.de}
\\ \vspace{4mm}{\large Manuel Hasert}\\
German Research School for Simulation Sciences GmbH and RWTH Aachen University \\
{\tt m.hasert@grs-sim.de}
\\ \vspace{4mm}{\large Sabine Roller}\\
German Research School for Simulation Sciences GmbH and RWTH Aachen University\\
{\tt s.roller@grs-sim.de}

\end{center}

\section*{Abstract}

Multi-scale problems are hard to tackle with rigorous
models, but some important engineering tasks are related to them.
But with increasing available computational power they get into reach of 
numerical simulations, which allows a better design process of technical devices,
for example to reduce generated noise.
In this work we present a strategy, based on the separation of spatial regions
with different numerical requirements.
This separation allows us to deploy the best-suited numerical methods in each
domain \cite{utzmann-2006}.
Such an approach also has the benefit to take advantage of specialized hardware
as shown in \cite{klimach-2009}. We now take this one step further and use a octree mesh discretization defined across the complete computational domain.
This allows for efficient determination of arbitrarily shaped coupling
interfaces in logarithmic complexity \cite{meagher-1982}. A key benefit of the globally known topology is the possibility to search the
neighborhood in parallel on distributed systems, with minimal requireddata from remote.
We therefore show a fully parallel, highly distributed framework based on such 
meshes, which enables the simulation of large multi-scale problems with coupled 
spatial domains on massively parallel computing systems.
We emphasize, that the octree is deployed throughout the complete processing
chain from the mesh generation over the solver to the parallel output for
visualization.
As described in \cite{tu-2006}, the consideration of the full simulation process 
is essential for large scale simulations to avoid bottlenecks.
In our simulation aero-acoustics for technical devices with complex geometry we show that this fully distributed data structure allows a
deployment on hundred thousand MPI processes.


\bibliographystyle{plain}
\begin{thebibliography}{10}

\bibitem{meagher-1982}
{\sc D. Meagher}. {Geometric Modeling Using Octree Encoding}. Computer Graphics and Image Processing 19(2): 129--147 (1982).



\bibitem{tu-2006}
{\sc T. Tu and H. Yu and L. Ramirez-Guzman and J. Bielak and O. Ghattas and K.-L. Ma and D. R. O'Hallaron}. {From Mesh Generation to Scientific Visualization: an End-to-End approach to parallel supercomputing}. SC'06, ISBN 0-7695-2700-0, ACM (2006).



\bibitem{utzmann-2006}
{\sc J. Utzmann and T. Schwartzkopff and C.-D. Munz and M. Dumbser}. {Heterogeneous Domain Decomposition for Computational Aeroacoustics}. AIAA Journal 44(10): 2231--2250 (2006).



\bibitem{klimach-2009}
{\sc H. Klimach and S. Roller and C.-D. Munz}. {Heterogeneous Parallelism of Aero-Acoustic Applications Using PACX-MPI}. Interdisciplinary Information Sciences 15(1): 79--83 (2009).

\end{thebibliography}
