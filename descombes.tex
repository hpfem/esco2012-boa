\title{High Order DGTD Method with Local Time Stepping for the Solution of the First Order Maxwell Equations}
\tocauthor{S. Descombes} \author{} \institute{}
\maketitle
\begin{center}
{\large St\'ephane Descombes}\\
Université Nice Sophia Antipolis and {\sc Nachos} project-team, INRIA Sophia Antipolis - M\'editerran\'ee research center\\
{\tt sdescomb@unice.fr}
\\ \vspace{4mm}{\large St\'ephane Lanteri, Jospeh Charles}\\
{\sc Nachos} project-team, INRIA Sophia Antipolis - M\'editerran\'ee  research center\\
{\tt Stephane.Lanteri@inria.fr, Joseph.Charles@inria.fr}
\\ \vspace{4mm}{\large Julien Diaz}\\
{\sc Magique-3D} project-team, INRIA Bordeaux - Sud-Ouest                research cente\\
{\tt Julien.Diaz@inria.fr}

\end{center}

\section*{Abstract}

In this talk we are interested in the discretization in space and time of the
Maxwell's equations in the presence of locally refined meshes.
A discontinuous Galerkin methods is
used and naturally allow high order spatial
approximation of the field in each cell.
The question is now the choice of the
time discretization. Most explicit time stepping methods are conditionally stable and
the finest element in a non-uniform mesh dictates the maximum time step allowed
to all the other elements of the computational domain. 
Various local explicit time stepping strategies have been then proposed,
presenting the advantage to use non-uniform time step sizes on non-uniform meshes to
further improve the efficiency of the numerical scheme by setting time steps accordingly to their corresponding element size.
In this talk, we propose fully explicit high order local time stepping strategies in the spirit of those
recently proposed in [1]. This is done here in the framework
of a non-dissipative discontinuous Galerkin time domain (DGTD) method for the solution of the first order form of the system of Maxwell's equations.

\bibliographystyle{plain}
\begin{thebibliography}{10}

\bibitem{1zz}
{\sc J. Diaz and M.J. Grote}. {Energy conserving explicit local time-stepping for second-order wave equations}. SIAM J. Sci. Comput. 31 (2009) 1985--2014.

\end{thebibliography}
