\title{An Efficient Dynamic $hp$-Discontinuous Galerkin Formulation for Time-Domain Electromagnetics}
\tocauthor{S. Schnepp} \author{} \institute{}
\maketitle
\begin{center}
{\large Sascha Schnepp}\\
Graduate School CE, TU Darmstadt\\
{\tt schnepp@gsc.tu-darmstadt.de}

\end{center}

\section*{Abstract}

The discontinuous Galerkin method (DGM) \cite{reed} received considerable development over the past two decades leading it to a mature state. In many cases the DGM is applied in a spectral-like manner on static meshes using a fixed approximation order throughout all elements. The strictly local support of the basis functions, however, renders the method highly suitable for adapting the local element size, $h$, as well as the local approximation order, $p$, in an element-wise fashion based on the local solution behavior.
In this talk a DG formulation on hexahedral meshes \cite{cohen} supporting dynamic $hp$-refinement \cite{houston} containing high-level hanging nodes and anisotropic refinement in both, $h$ and $p$, is presented. During the development special care of the computational efficiency of the basic algorithm and the adaptation procedures has been taken. The method is especially successful for multi-scale problems, where the adaptive approach leads to significant savings in computational resources and time.

\bibliographystyle{plain}
\begin{thebibliography}{10}

\bibitem{reed}
{\sc W. Reed and T. Hill}. {Triangular mesh methods for the neutron transport equation}. Tech. rep., Los Alamos Scientific Laboratory Report (1973).



\bibitem{cohen}
{\sc G. Cohen and X. Ferrieres and S. Pernet}. {A spatial high-order hexahedral discontinuous Galerkin method to solve Maxwell's equations in time domain}. J. Comput. Phys. 217 (2) (2006) 340-363.



\bibitem{houston}
{\sc P. Houston and E. S\"uli}. {A note on the design of hp-adaptive finite element methods for elliptic partial differential equations}. Comput. Method Appl. M 194 10 (2-5) (2005) 229-243.

\end{thebibliography}
