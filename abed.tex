\title{New Quadratic Solid-Shell Elements and their Evaluation on Popular Benchmark Problems}
\tocauthor{F. Abed-Meraim} \author{} \institute{}
\maketitle
\begin{center}
{\large \underline{Farid Abed-Meraim}}\\
LEM3 UMR CNRS 7239, Arts et Metiers ParisTech, 4 rue Augustin Fresnel, 57078 Metz\\
{\tt farid.abed-meraim@ensam.eu}
\\ \vspace{4mm}{\large Vuong-Dieu Trinh}\\
LEM3 UMR CNRS 7239, Arts et Metiers ParisTech, 4 rue Augustin Fresnel, 57078 Metz\\
{\tt vuong-dieu.trinh@metz.ensam.fr}
\\ \vspace{4mm}{\large Alain Combescure}\\
LaMCoS UMR CNRS 5259, INSA-Lyon, 18, 20 rue des Sciences, 69621 Villeurbanne\\
{\tt alain.combescur@insa.lyon.fr}

\end{center}

\section*{Abstract}

In recent years, considerable effort has been devoted to the development of 3D finite elements able to model thin structures (Cho et al., 1998; Sze and Yao, 2000; Abed-Meraim and Combescure, 2002; Vu-Quoc and Tan, 2003; Chen and Wu, 2004). To this end, coupling solid and shell formulations proved to be an interesting strategy, providing continuum finite element models that can be efficiently used for structural applications.

In the present work, two solid-shell elements are formulated (a 20-node and a 15-node element) based on a purely three-dimensional approach. The advantages of these elements are shown through the analysis of various structural problems. Note that their main advantage is to allow complex structural shapes to be simulated without classical problems of connecting zones meshed with different element types. These solid-shell elements have a special direction called the ''thickness'', along which a set of integration points are located. Reduced integration is also used to prevent some locking phenomena and to increase computational efficiency.

Focus will be placed here on linear benchmark problems, where it is shown that these solid-shell elements perform much better than their counterparts, conventional solid elements.

\bibliographystyle{plain}
\begin{thebibliography}{10}

\bibitem{Cho-Park-Lee}
{\sc C. Cho and H.C. Park and S.W. Lee}. {Stability analysis using a geometrically nonlinear assumed strain solid shell element model}. Finite Elem. Analysis Des. 29 (1998) 121-135.



\bibitem{Sze-Yao}
{\sc K.Y. Sze and L.Q. Yao}. {A hybrid stress ANS solid-shell element and its generalization for smart structure modeling. Part I-solid-shell element formulation}. Int. J. Num. Meth. Eng. 48 (2000) 545-564.



\bibitem{Abed-Meraim-Combescure}
{\sc F. Abed-Meraim and A. Combescure}. {SHB8PS - a new adaptive, assumed-strain continuum mechanics shell element for impact analysis}. Comp. and Struct. 80 (2002) 791-803.



\bibitem{Vu-Quoc-Tan}
{\sc L. Vu-Quoc and X.G. Tan }. {Optimal solid shells for non-linear analyses of multilayer composites. I. Statics}. Comp. Meth. Applied Mech. Eng. 192 (2003) 975-1016.



\bibitem{Chen-Wu}
{\sc Y.I. Chen and G.Y. Wu}. {A mixed 8-node hexahedral element based on the Hu-Washizu principle and the field extrapolation technique}. Structural Engineering and Mechanics 17 (2004) 113-140.

\end{thebibliography}
