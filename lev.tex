\title{The Mathematical Model of Experimental Sensor for Detecting of Plant Material Distribution on the Conveyor}
\tocauthor{J. Lev} \author{} \institute{}
\maketitle
\begin{center}
{\large \underline{Jakub Lev}}\\
Czech University of Life Sciences, Prague\\
{\tt jlev@tf.czu.cz}
\\ \vspace{4mm}{\large Petr Mayer}\\
Czech Technical University in Prague\\
{\tt pmayer@mat.fsv.cvut.cz}
\\ \vspace{4mm}{\large Vaclav Prosek}\\
Czech University of Life Sciences, Prague\\
{\tt prosek@tf.czu.cz}
\\ \vspace{4mm}{\large Marie Wohlmuthova}\\
Czech University of Life Sciences, Prague\\
{\tt wohlmuth@tf.czu.cz}

\end{center}

\section*{Abstract}

Electrical capacitance tomography (ECT) is used to image cross-sections of industrial processes containing dielectric material. A lot of articles have been published about this technique~\cite{xie_et_al-1,yang-2}. Two types of sensors are often used: sensors with the circular imaging area and sensors with square imaging area. However, for some practical applications (i. e. conveyor belt) the sensor with a rectangular imaging area can be suitable. In this article the mathematical model of segmented capacity sensor (SCS) is described and verified by the measurements. During the creation of ECT sensors mathematical models, electrostatic field is often considered. This field can be described by the Laplace equation~\cite{guo_et_al-3,yang_and_Liu-4}. The mathematical model of ECT sensor is suitable for high-quality sensor design and it is often used to create sensitivity maps. Those maps are important for an image reconstruction process. It has been proved by the measurement that SCS electric field can not be simplified to the electrostatic field. Therefore, in this article the complex mathematical model creation which includes other physical factors is described. Agros2D and Hermes2D software for solutions of physical fields were used for SCS mathematical model creation.

\bibliographystyle{plain}
\begin{thebibliography}{10}

\bibitem{xie_et_al-1}
{\sc C. G. Xie and S. M. Huang and B. S. Hoyle and R. Thorn and C. Lenn and D. Snowden and M. S. Beck}. {Electrical capacitance tomography for flow imaging: system model for the development of image reconstruction algorithms and design of primary sensors}. IEE Proceedings-G 139 (1992) 89-98.



\bibitem{yang-2}
{\sc W. Yang}. {Design of electrical capacitance tomography sensors}. Meas. Sci. Technol. 21 (2010) 1-13.



\bibitem{guo_et_al-3}
{\sc Z. Guo and F. Shao and D. Lv}. {Sensitivity matrix construction for electrical capacitance tomography based on the difference model}. Flow Measurement and Instrumentation 20 (2009) 95-102.



\bibitem{yang_and_Liu-4}
{\sc W. Q. Yang and S. Liu}. {Electrical capacitance tomography with a Square Sensor}. In: Proceedings of the 1st World Congress on Industrial Processes Tomography, 14-17 April 1999, pp. 313-317, Buxton, UK.

\end{thebibliography}
