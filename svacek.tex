\title{On Mathematical Modeling of Flow Induced Vibrations of an Airfoil with Three Degrees of Freedom}
\tocauthor{P. Svacek} \author{} \institute{}
\maketitle
\begin{center}
{\large Petr Sv\' a\v cek}\\
Czech Technical University in Prague, Faculty of Mechanical Engineering, Department of Technical Mathematics, Karlovo nam. 13, Praha 2\\
{\tt Petr.Svacek@fs.cvut.cz}

\end{center}

\section*{Abstract}

The subject of the paper is the numerical simulation of the interaction of two-dimensional incompressible viscous flow and a vibrating airfoil with large amplitudes. The airfoil with
three degrees of freedom performs rotation around an elastic axis, oscillations in the vertical direction and rotation of a flap.
The nonlinear effects are taken into account with the aid of nonlinear stiffness coefficients. The numerical simulation consists of the finite element solution of the Navier-Stokes equations for the laminar model or the Reynolds averaged Navier-Stokes equations combined with model of turbulence. The one or two equations models are considered. The flow models are coupled with a system of nonlinear ordinary differential equations describing the airfoil motion. The time-dependent computational domain and a moving grid are treated by the Arbitrary Lagrangian-Eulerian formulation of the flow equations. High Reynolds numbers require the
application of a suitable stabilization of the finite element discretization. The developed method is used for the computation of flow-induced oscillations of the airfoil near the flutter instability, when the displacements of the airfoil are large. The numerical results obtained with the aid of  laminar and turbulent models are compared. 

\bibliographystyle{plain}
\begin{thebibliography}{10}

\bibitem{Horacek1}
{\sc J. Hor\' a\v cek and P. Sv\' a\v cek and M. Feistauer}. {Contribution to finite element modelling of airfoil aeroelastic instabilities}. Applied and Computational Mechanisc 1, 2007,  43--52.



\bibitem{Svacek2}
{\sc P. Sv\' a\v cek and M. Feistauer and J. Hor\' a\v cek}. {Numerical   simulation of flow induced airfoil vibrations with large amplitudes.}. Journal of Fluids and Structure \textbf{23} (2007), 391--411.

\end{thebibliography}
