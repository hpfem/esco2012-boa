\title{Mathematical Modeling of Morphology Evolution in Heterophase Polymers}
\tocauthor{M. Vonka} \author{} \institute{}
\maketitle
\begin{center}
{\large \underline{Michal Vonka}}\\
Institute of Chemical Technology Prague\\
{\tt vonkai@vscht.cz}
\\ \vspace{4mm}{\large Juraj Kosek}\\
Institute of Chemical Technology Prague\\
{\tt jkk@vscht.cz}

\end{center}

\section*{Abstract}

The contribution shows a predictive modeling of polymeric systems which undergo phase separation during the morphogenesis. The approach considers the production of high-impact polystyrene (HIPS) as a representative example. The complex process of morphology evolution may be roughly divided into these stages: (1) miscible system of two components (e.g., styrene and polybutadiene), (2) phase separation by nucleation (if seeds are present) or spinodal decomposition into two phase system, (3) phase inversion causing the continuous phase to become dispersed and vice versa, (4) further morphology evolution involving the coarsening, coagulation, etc. 

The model capable of simulating the evolution of heterophase polymeric structures is based upon Cahn-Hilliard approach, which considers the gradients of chemical potential evaluated from the Flory-Huggins equation as the driving forces of species transport. Furthermore, Landau-Ginzburg functional introduces the local approximation of the surface tension between immiscible phases.  Similar approach was implemented by Nauman [1]. The dynamic model is used to classify the importance of different phenomena during the above described stages of morphology evolution. The experimentally observed mechanisms [2] of the creation of the so-called salami morphology were modeled. This type of morphology, sometimes called a double emulsion, has a significant impact on the material properties, mostly on mechanical resistance and gloss. Deeper understanding of the morphology-ruling phenomena during each stage of the evolution allows to optimize HIPS morphology and properties of the resulting material. At the current state, the model gives a possibility to predict morphology evolution in a batch system. The model is equipped with approximate description of polymerization, effect of agitator, presence of grafted copolymer and others. An insight into phenomena like phase separation, coagulation of droplets of rubbery phase and Ostwald ripening of particles with high surface energies is possible with our model. Because the coalescence and coagulation of emulsions is highly dependent on flow patterns in used devices, the description of hydrodynamic effects on the salami morphology evolution is going to be introduced as well as the description of phase separation processes in series of continuously mixed reactors. 


\bibliographystyle{plain}
\begin{thebibliography}{10}

\bibitem{Alfaraj-Naumann}
{\sc A. Alfaraj and E.B. Nauman}. {Spinodal decomposition in ternary systems with significantly different component diffusivities}. Macromolecular theory and simulations 16 (2007) 627-631.



\bibitem{Leal-Asua}
{\sc P. G. Leal and J. M. Asua}. {Evolution of the morphology of HIPS particles}. Polymer, 50 (2009) 68-76.

\end{thebibliography}
