\title{Stepsize and Order Control in Ordinary Differential Equation Solvers}
\tocauthor{J. Butcher} \author{} \institute{}
\maketitle
\begin{center}
{\large John Butcher}\\
University of Auckland\\
{\tt butcher@math.auckland.ac.nz}

\end{center}

\section*{Abstract}

When a step of size $h$ of a differential equation solver is performed, along with the approximation
to the next solution value, is a quantity which is asymptotically proportional to $h^{q+1}$.
This is used to represent the local truncation error for stepsize control purposes.
As a result of comparing the error estimate with the user-supplied tolerance, a decision
is made whether to accept the step or not and what the new stepsize should be.
Traditionally the new step is taken to be something like
\[
h \times \max\left(0.5, \min\left(2.0, 0.9\left(\frac{\mbox{\small tolerance}}{\mbox{\small estimate}}\right)^{1/(q+1)}\right)\right),
\]
where the ``magic numbers" $0.5$, $2.0$ and $0.9$ are somwhat arbitrary safety factors.
In recent years the choice of the new stepsize has usually been modified according to the principles of PI-control [2].

In an attempt to put these ideas on a rational basis, a new approach has been proposed [1].  This makes it possible 
to switch also between stiff and non-stiff options and between different orders in a systematic way.
The principle is to attempt to minimize 
\[
\{\mbox{rate of error production}\} + T\times \{\mbox{rate of  work expenditure}\}.
\]
The Lagrange multiplier $T$ plays a similar role to the user-supplied tolerance and expresses a request to emphasise accuracy versus cost depending on the value of $T$.


\bibliographystyle{plain}
\begin{thebibliography}{10}

\bibitem{buu1}
{\sc J. C. Butcher}. {Order, stepsize and stiffness switching}. Computing 44, (1990) 209-220.



\bibitem{buu2}
{\sc K. Gustafsson and M. Lundh and G. S\"oderlind}. {A PI stepsize control for the numerical solution of ordinary differential equations}. BIT 28 (1988) 270-287.

\end{thebibliography}
