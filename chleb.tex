\title{On an Optimal Node and Degree Distribution in the $hp$-FEM}
\tocauthor{J. Chleboun} \author{} \institute{}
\maketitle
\begin{center}
{\large Jan Chleboun}\\
Czech Technical University, Faculty of Civil Engineering, Prague\\
{\tt chleboun@mat.fsv.cvut.cz}

\end{center}

\section*{Abstract}

The accuracy of an $hp$-FE approximation of a solution
to a weakly formulated problem originating from differential equations
can be controlled by both the position of mesh nodes and
by the maximum degree of (polynomial) basis functions forming the 
finite-dimensional space where the approximate solution
is searched for, see \cite{BG,Sch,SSD,DKP}. The degree need not be uniformly distributed
over the set of mesh elements. In fact, nonuniformly distributed
degree of basis functions is desirable because it allows 
to use high-degree (that is, more complex) functions at areas
where they substantially contribute to the accuracy of 
the approximate solution whereas the other part of the
solution is sufficiently accurately captured by low-degree basis functions.
 
Although the $hp$-FEM flexibility in controlling the error
of an approximate solution is one of the strong points of the method,
it also causes difficulties. Indeed, unlike the $h$-FEM adaptivity
where only the mesh is modified, adaptive $hp$-FEM algorithms have
to solve the problem what combination of a mesh modification and polynomial
degree redistribution to choose. 

To better understand the approximation potential of the $hp$-FEM,
the following goal has been set: given a fixed number of degrees
of freedom (DOF), find its optimal distribution between the mesh
and the basis functions, and optimize the position of the mesh nodes. 
The difference between the known exact solution and the calculated 
$hp$-FEM solution
measured by the $H^1$-norm serves as the optimization criterion.
Even if the underlying problem is only a 1D elliptic boundary
value problem, the entire optimization problem is rather demanding
due to its combinatorial character. The number of possible degree
distributions increases as fast as $2^n$ where $n$ is equal to
the DOF. Each polynomial degree distribution determines the number
of mesh nodes; their position is afterwards optimized.
The problem of the optimal DOF utilization has been solved numerically
in the Matlab environment.




\bibliographystyle{plain}
\begin{thebibliography}{10}

\bibitem{BG}
{\sc I. Babu\v{s}ka and B.Q. Guo}. {The $h$, $p$ and $h-p$ version of the finite element method: basis theory and applications}. Advances in Engineering Software, Volume 15, Issue 3-4, 1992.



\bibitem{Sch}
{\sc C. Schwab}. {$p$- and $hp$- Finite Element Methods: Theory and Applications in Solid and Fluid Mechanics}. Oxford University Press, 1998.



\bibitem{SSD}
{\sc P. \v{S}ol\'{\i}n and K. Segeth and I. Dole\v{z}el}. {Higher-Order Finite Element Methods}. Chapman \&\ Hall/CRC Press, 2003.



\bibitem{DKP}
{\sc L. Demkowicz and J. Kurtz and D. Pardo and W. Rachowicz and M. Paszynski and A. Zdunek}. {Computing with $hp$-Adaptive Finite Elements}. Chapman \&\ Hall/CRC Press, 2007.

\end{thebibliography}
