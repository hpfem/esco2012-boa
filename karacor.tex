\title{A Predictive Artificial Neural Network Model for Distribution of Aquatic Microbial Community in Kucukcekmece Lagoon Turkey}
\author{} \tocauthor{A.G. Karacor}\institute{}
\maketitle
\begin{center}
{\large \underline{Adil Gursel Karacor}}\\
Turkish Air Force\\
{\tt karacor@gmail.com}
\\ \vspace{4mm}{\large Nuket Sivri}\\
Istanbul University\\
{\tt nuket@istanbul.edu.tr}

\end{center}

\section*{Abstract}

Lagoon temperature determines the rate of the decomposition of organic matter and the saturation concentration of dissolved oxygen, hence affects the distribution of microbial community in water. Combined with industrial waste, water temperature becomes a crucial parameter. Therefore, estimation of maximum water temperature becomes very important, especially during summertime when high temperatures may pose danger for the habitat of lagoons. Quite a few feed forward artificial neural network models were developed to predict the maximum water temperatures of the Kucukcekmece Lagoon for the five days ahead. Satisfactory results were achieved as the average prediction error turned out to be only a little under 2 degrees celsius.

\bibliographystyle{plain}
\begin{thebibliography}{10}

\bibitem{Boyee}
{\sc K.D. Boyee}. {A Guide To Stream Habitat Analysis Using The Instream Incremental Methodology}. Instream Flow Information Paper 12. U.S. Fish and Wildlife Service EWS/OBS 82/26 (1982) 248.



\bibitem{Rounds}
{\sc S.A. Rounds}. {Development Of Neural Network Model For Dissolved Oxygen In The Tualantin River Oregon}. Proceedings Of Second Federal Interagency Hydrologic Modelling Conference, (2002), Las Vegas, Nevada.



\bibitem{Aksoy}
{\sc H. Aksoy and A. Dahamsheh}. {Artificial Neural Network Models For Forecasting Monthly Precipitation In Jordan}. Stochastic Environmental Research And Risk Assessment, Vol. 23 No 7 (2008) 917-931.

\end{thebibliography}
