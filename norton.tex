\title{Higher Order Splitting Methods for Semi-Groups}
\tocauthor{R. Norton} \author{} \institute{}
\maketitle
\begin{center}
{\large Richard Norton}\\
La Trobe University\\
{\tt Richard.Norton@latrobe.edu.au}

\end{center}

\section*{Abstract}

We consider splitting methods for solving ODEs.  Splitting methods with real coefficients in the composition require at least one negative time step to achieve order of greater than 2 \cite{blanes05}.  However, a negative time step cannot be used when the integrators form a semi-group (eg. most dissipative systems).  We present one approach for circumventing this difficulty and give applications that preserve volume contraction or a Lyapunov function.  This is joint work with Blanes, Casas, Bader, Quispel and McLaren.

\bibliographystyle{plain}
\begin{thebibliography}{10}

\bibitem{blanes05}
{\sc S. Blanes and F. Casas}. {On the necessity of negative coefficients for operator splitting schemes of order higher than two}. Appl. Num. Math. 54 (2005) 23-37.

\end{thebibliography}
