\title{Hybrid Parallel Solvers for Finite Element Approximation of PDEs}
\author{} \tocauthor{J. Principe} \institute{}
\maketitle
\begin{center}
{\large Santiago Badia}\\
Centre Internacional de M\`etodes Num\`erics en Enginyeria (CIMNE), Universitat Polit\`ecnica de Catalunya\\
{\tt sbadia@cimne.upc.edu}
\\ \vspace{4mm}{\large Alberto F. Mart\'in}\\
Centre Internacional de M\`etodes Num\`erics en Enginyeria (CIMNE), Universitat Polit\`ecnica de Catalunya\\
{\tt amartin@cimne.upc.edu}
\\ \vspace{4mm}{\large \underline{Javier Pr\'incipe}}\\
Centre Internacional de M\`etodes Num\`erics en Enginyeria (CIMNE), Universitat Polit\`ecnica de Catalunya\\
{\tt principe@cimne.upc.edu}

\end{center}

\section*{Abstract}

Future increases of the computational power of  
distributed-memory architectures will 
most likely be achieved with a substantially higher degree 
of multicore parallelism per node, which
in turn will be accompanied with more complex hierarchical
cache/memory designs. On the algorithmic level,
hybrid algorithms and software appear to be best-suited 
for the hierarchical organization
of the underlying parallel hardware~\cite{RBH}. 
Under this scenario, domain decomposition (DD) techniques provide a natural 
framework for the development of hybrid parallel solvers 
tailored for FE analysis. In this work, we consider message-passing (MPI) scalable DD algorithms with 
coarse solver combined with several multi-threaded subdomain solvers, comprising
multi-threaded sparse direct solvers (e.g., PARDISO) as well as a recursive/nested
approaches based on multi-threaded DD subdomain solvers (implemented in OpenMP).
This software machinery allows to leverage the two levels of hardware parallelism
(inter-node and intra-node) in several ways, ranging from a pure one-level MPI approach, 
in which we have as many subdomains/MPI ranks as cores, to a hybrid approach, in which we
have one subdomain/MPI rank per node/socket, and as many threads as cores in a node/socket.
Extensive numerical experiments are performed in order to evaluate which is the best
setting in every case, depending on subdomain problem size, number of nodes and cores. Several 
DD techniques and coarse solvers will be discussed.

\bibliographystyle{plain}
\begin{thebibliography}{10}

\bibitem{RBH}
{\sc S. Rajamanickam and E. G. Boman and M. A. Heroux}. {ShyLU: A Hybrid-Hybrid Solver for Multicore Platforms}. Technical Report. Available on-line at {\tt http://www.cise.ufl.edu/{\raise.17ex\hbox{$\scriptstyle\mathtt{\sim}$}}srajaman/papers/ShyLU.pdf}.

\end{thebibliography}
