
\title{How to Metrize Chains: A Numerical Study}
\author{} \tocauthor{A. DiCarlo} \institute{}
\maketitle
\begin{center}
{\large \underline{Antonio DiCarlo}}\\
University Roma Tre\\
{\tt adicarlo@mac.com}
\\ \vspace{4mm}{\large Stefano Gabriele}\\
University Roma Tre\\
{\tt gabriele@uniroma3.it}
\\ \vspace{4mm}{\large Valerio Varano}\\
University Roma Tre\\
{\tt v.varano@uniroma3.it}

\end{center}

\section*{Abstract}

The concept of \emph{metrized chains} was introduced by DiCarlo et al. \cite{adc&09} to establish a discrete metric structure on top of the discrete measure-theoretic structure embodied in the complex of \emph{measured chains} built on an underlying \emph{cell complex}. \par
%
A (finite) cell complex is an equivalence class of (geometrically different) meshes sharing the same topology, i.e., the same sets of cells of each dimension and the same incidence relations. Measured chains attach a signed $k$-measure to $k$-cells---i.e., length to 1-cells, area to 2-cells, volume to 3-cells, and so on---, thus restoring part of the geometrical information stripped away by the purely topological construction of a cell complex. Measured cochains---introduced by duality---represent densities with respect to the measure imparted to cells by measured chains, the chain-cochain pairing being just discrete integration. \par
%
Measure does not exhaust geometry. To reintroduce metric properties, chains have to be endowed with an inner product. Choosing an inner product is tantamount to identifying chains with cochains. Via this identification, boundary and coboundary operators---acting respectively on chains and cochains---may be composed with each other, giving rise to Laplace-deRham operators. Changing the inner product changes neither the boundary nor the coboundary operators. However, it does change the hierarchies of Hodge and Laplace-deRham operators. To establish a direct link between the metric structure imparted to chains and that of the physical manifold being approximated, $k$-chains are mapped to $k$-vector fields on this manifold, and the inner product between two chains is identified with the inner product between the corresponding multivector fields. \par
% 
A chain can be mapped to a multivector field in infinitely many legitimate ways. In this study, we start exploring how different choices may affect the computation of basically important discrete operators such as Hodge and Laplace-deRham.


\bibliographystyle{plain}
\begin{thebibliography}{10}

\bibitem{adc&09}
{\sc A. DiCarlo and F. Milicchio and A. Paoluzzi and V. Shapiro}. {Discrete physics using metrized chains}. In: W.F.~Bronsvoort et al., eds., \textit{Proc. SPM'09 -- 2009 SIAM/ACM Joint Conference on Geometric and Physical Modeling}, ACM, pp.~135--145, 2009 \texttt{[doi:10.1145/1629255.1629273]}.

\end{thebibliography}
