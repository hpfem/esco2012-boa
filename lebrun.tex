\title{Simulation of Thermo-Mechanical Contact Between Fuel Pellets and Cladding in UO$_2$ Nuclear Fuel Rods}
\tocauthor{D. Lebrun-Grandie} \author{} \institute{}
\maketitle
\begin{center}
{\large \underline{Damien Lebrun-Grandie}, Jean Ragusa}\\
Texas A\&M University\\
{\tt dalg24@ne.tamu.edu, ragusa@ne.tamu.edu}

\end{center}

\section*{Abstract}

As fission process heats up nuclear fuel, its UO$_2$ pellets, stacked atop of each other, swell both radially and axially, while the surrounding Zircaloy cladding creeps, so that cladding and pellet eventually come into contact.
This exacerbates chemical degradation of the protective cladding, further enables the propagation of cracks and, thus, threatens the integrity of the clad \cite{Aas1972}.
Along these lines, pellet-cladding interaction is a major concern in fuel rod design and reactor core operation in light water reactors.
Accurately modeling fuel behavior is a challenging task because the mechanical contact problem strongly depends on temperature distribution,
and the coupled pellet-cladding heat transfer problem, in turn, is affected by changes in geometry induced by body deformations and stresses generated at the contact interface.
Previous approaches either focused on mechanics and only loosely coupled to the thermal problem (operator split) \cite{Denis2003},
or, although handling coupling between the various physics components in a consistent way, yet ignored pellet-cladding mechanical interaction \cite{Mihaila2009,Newman2009}.
We use Lagrange multipliers to enforce the non-penetration condition (in a weak sense).  Lagrange multipliers account for reaction forces between the bodies involved in contact.
The multibody thermo-mechanical contact problem is tackled using modern numerics, with higher-order finite elements and a Newton-based monolithic strategy \cite{Solin2010} to handle both nonlinearities
(coming from the non-linearity of the contact condition but as well as from the temperature-dependence of the fuel thermal conductivity for instance)
and coupling between the various physics components (gap conductance dependent upon the clad- pellet distance, thermal expansion coefficient or Young's modulus affected by temperature changes, etc.).
We will provide different numerical examples for one- and multiple-body contact problems to demonstrate the performance of the method.

\bibliographystyle{plain}
\begin{thebibliography}{10}

\bibitem{Aas1972}
{\sc S. Aas}. {Mechanical interaction between fuel and cladding}. Nuclear Engineering and Design, 21(2):237--253, 1972.



\bibitem{Denis2003}
{\sc A. Denis and A. Soba}. {Simulation of pellet-cladding thermomechanical interaction and fission gas release}. Nuclear Engineering and Design, 223(2):211--229, 2003.



\bibitem{Mihaila2009}
{\sc B. Mihaila and M. Stan and J. Ramirez and A. Zubelewicz and and P. Cristea}. {Simulations of coupled heat transport, oxygen diffusion, and thermal expansion in UO$_2$ nuclear fuel elements}. Journal of Nuclear Materials, 394(2-3):182--189, 2009.



\bibitem{Newman2009}
{\sc C. Newman and G. Hansen and and D. Gaston}. {Three dimensional coupled simulation of thermomechanics, heat, and oxygen diffusion in UO$_2$ nuclear fuel rods}. Journal of Nuclear Materials, 392(1):6--15, 2009.



\bibitem{Solin2010}
{\sc P. Solin and J. Cerveny and L. Dubcova and and D. Andrs}. {Monolithic discretization of linear thermoelasticity problems via adaptive multimesh $hp$-fem}. Journal of Computational and Applied Mathematics, 234(7):2350--2357, 2010.


\end{thebibliography}
