\title{Adaptive Finite Element Solution of Second-Order Approximations of Neutron Transport}
\tocauthor{M. Hanus} \author{} \institute{}
\maketitle
\begin{center}
{\large Milan Hanu{\v s}}\\
University of West Bohemia in Pilsen\\
{\tt mhanus@kma.zcu.cz}

\end{center}

\section*{Abstract}

In this talk, we will be interested in computational modeling of processes induced by neutral particles, relevant to economical and safe operation of nuclear facilities. 

The general mathematical model of radiative transfer of energy by elementary particles is provided by the linear form of the Boltzmann's transport equation. This first-order PDE with integral terms describes dependence of the radiation field on six phase-space variables (position, direction and energy of the transported particles). 

High-dimensionality and complicated structure of the equation for real-world problems preclude analytical solution and present serious difficulties when a full-scale computational solution is attempted. Various dimension-reduction techniques are therefore usually employed along with appropriate numerical solution schemes. In this talk, we will be particularly interested in approximations allowing to transform the governing equation into a second-order PDE (or system of such equations). Advanced, well-developed numerical methods for elliptic PDEs can then be used to obtain the solution. 

The diffusion and the so called simplified $P_N$ approximations are some of the widely used approximations from this class. It has been pointed out (\cite{Larsen2,Larsen4}) recently, that by using cleverly precomputed parameters in the associated equations, these approximations can quite accurately capture selected transport phenomena while being much cheaper to solve than the original first-order transport equation. 

In this talk, we will present a general framework for solving diffusion and simplified $P_N$ equations, built upon the hp-FEM library Hermes (\cite{Hermes}). Use of this framework will be demonstrated on several neutron transport problems and performance of various h- and hp-adaptivity options available in Hermes will be discussed. Special tuning of these adaptivity techniques for the aforementioned transport approximations will also be covered. Finally, comparing the results with those from a recently published article (\cite{Ragusa1}) on the implementation of the simplified $P_3$, hp-FE approximation in an older version of the Hermes library may also reveal the progress made in the development of the library itself.

\bibliographystyle{plain}
\begin{thebibliography}{10}

\bibitem{Larsen2}
{\sc E. W. Larsen}. {Asymptotic Diffusion and Simplified PN Approximations for Diffusive and Deep Penetration Problems. Part 1: Theory}. Transp. Theory Stat. Phys. 39 (2011), 110-163.



\bibitem{Larsen4}
{\sc E. W. Larsen}. {2-D Anisotropic Diffusion in Optically Thin Channels}. Trans. Am. Nucl. Soc. 101 (2009), 387.



\bibitem{Ragusa1}
{\sc J. C. Ragusa}. {Application of h-, p-, and hp-mesh adaptation techniques to the SP3 equations}. Transp. Theory Stat. Phys. 39 (2011), 234-254.



\bibitem{Hermes}
{\sc P. Solin et al.}. {Hermes - Higher-Order Modular Finite Element System (User's Guide)}. url: http://hpfem.org/.

\end{thebibliography}
