\title{Modeling of Degradation Processes in Historical Mortars}
\tocauthor{J. Sykora} \author{} \institute{}
\maketitle
\begin{center}
{\large Jan S\'{y}kora}\\
Czech Technical University in Prague, Faculty of Civil Engineering, Department of Mechanics\\
{\tt jan.sykora.1@fsv.cvut.cz}

\end{center}

\section*{Abstract}

The aim of presented paper is modeling degradation processes in
the historical mortars exposed to water and salt impact. Internal
damage caused by ice and salt crystallization in pores is one of
the most important factors limiting the service life of historical
structures. To quantify the internal damage, complex
thermo-hygro-mechanical model is introduced based on penetration
of crystals through the pore structure~\cite{Scherer} and damage mechanics~\cite{Liu}. An example of two-dimensional water transport in the
environment with temperature below freezing point in mortar will
be presented to support the theoretical derivations.

\bibliographystyle{plain}
\begin{thebibliography}{10}

\bibitem{Scherer}
{\sc G. W. Scherer}. {Crystallization in pores}. Cement and Concrete Research 29 (1999) 1347-1358.



\bibitem{Liu}
{\sc L. Liu and G. Ye and E. Schlagen and H. Chen and Z. Qian and W. Sun and K. van Breugel}. {Modeling of the internal damage of saturated cement paste due to ice crystallization pressure during freezing}. ement and Concrete Research 33(5)  (2011) 562-571.

\end{thebibliography}
