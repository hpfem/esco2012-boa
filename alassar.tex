\title{Transient Heat Conduction in a Spherical Annulus with Eigenvalue-Dependent Boundary Conditions}
\tocauthor{R. Alassar} \author{} \institute{}
\maketitle
\begin{center}
{\large Rajai Alassar}\\
King Fahd University of Petroleum and Minerals\\
{\tt alassar@kfupm.edu.sa}

\end{center}

\section*{Abstract}

The External Problem of heat conduction in the annulus between two concentric spheres is solved. The External Problem means that the temperature in the inner sphere is assumed to be unsteady but spatially uniform, while the temperature on the surface of the outer sphere is maintained at a constant value. A preliminary investigation shows that the External Problem, usually considered in the literature of heat convection from spheres to simplify the original Internal Problem (where the temperature in the inner sphere is neither steady nor uniform), turns out to be mathematically involved. The External Problem gives rise to a boundary value problem with boundary conditions that depend on the eigenvalues; a non-standard Sturm-Liouville problem. The technique developed by Behrndt, and Binding et al. will be used to solve the problem. In summary, the eigenfunctions of the present problem, on their own, are not necessarily orthogonal in L2 but in the larger space L2+C2 with a suitable, in this case Pontryagin space inner product, the eigenvectors of the operator generated by this problem are orthogonal.

\bibliographystyle{plain}
\begin{thebibliography}{10}

\bibitem{"Behrndt"}
{\sc J. Behrndt}. {Boundary value problems with eigenvalue depending boundary conditions}. Math. Nachr. 282, No. 5, (2009) 659   689.



\bibitem{"Binding"}
{\sc P. A. Binding and P. J. Browne and B. A. Watson}. {Sturm Liouville problems with boundary conditions rationally dependent on the eigenparameter}. I. Proc. Edinb. Math. Soc. 45, (2002) 631  645.

\end{thebibliography}
