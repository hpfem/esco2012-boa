\title{Towards New Polymer Foam Insulation Material by the Use of Mathematical Modeling}
\tocauthor{R. Pokorny} \author{} \institute{}
\maketitle
\begin{center}
{\large \underline{Richard Pokorny}}\\
ICT Prague, Dept. of Chemical Engineering, Technicka 5, 166 28, Prague 6, Czech Republic\\
{\tt Richard.Pokorny@vscht.cz}
\\ \vspace{4mm}{\large Pavel Ferkl}\\
ICT Prague, Dept. of Chemical Engineering, Technicka 5, 166 28, Prague 6, Czech Republic\\
{\tt ferklp@vscht.cz}
\\ \vspace{4mm}{\large Kosek Juraj}\\
ICT Prague, Dept. of Chemical Engineering, Technicka 5, 166 28, Prague 6, Czech Republic\\
{\tt Juraj.Kosek@vscht.cz}

\end{center}

\section*{Abstract}

Some modern micro- and nano-structured materials (e.g., polymer
nano-foams) represent an industrially important class of materials
in today's technology, as they are widely used as heat insulation
materials due to their low thermal conductivity and light weight.
This contribution is thus concerned with two main topics: (i) the
development of the algorithm which is able to solve the coupled
conduction-radiation heat transfer problem on very fine
discretization grid, and (ii) the study and modeling of various
heat transfer phenomena which have to be considered at nano-scale.
The algorithm of Voronoi tessellation was used for the digital
reconstruction of the spatially two and three dimensional media
with the morphology of the polymer foam. Finite difference (FDM),
finite volume (FVM) and finite element method (FEM) were used for
the processing of heat transfer equations. We also report on the
experience with the new algebraic multigrid method, which is used
for the solution of large systems of algebraic equations. This
multigrid method, based on the aggregation of unknowns, allowed us
to calculate the heat conduction on cubic domains containing up to
 $500 \times 500 \times 500$ voxels (finite volumes) in 3D and
the coupled conduction-radiation problem on $2000 \times 2000$
pixels (in 2D). The concepts of both the gray medium and the
spectrally-dependent medium are employed in the modeling of heat
radiation. Intuitively, we expect that the effective thermal
conductivity of foam decreases with increasing porosity, because
polymers have about ten times higher thermal conductivity than
gases. However, the higher porosity also means less polymer phase,
which acts as a barrier for the thermal radiation. The combined
effect of conduction-radiation thus results in the minimum
effective thermal conductivity observed on the thermal
conductivity versus porosity curve. In common heat insulations
about 30\% of heat is transferred by radiation.


\bibliographystyle{plain}
\begin{thebibliography}{10}

\bibitem{Notay}
{\sc Notay Y}. {An aggregation-based algebraic multigrid method}. Tech. Rep. GANMN 08-02, ULB, Belgium, 2008.



\bibitem{Wang}
{\sc Wang M. and Pan N}. {Prediction of effective physical properties of complex multiphase materials}. Mat. Sci. Eng. R., 63:1 (2008). doi: 10.1016/j.mser.2008.07.001.

\end{thebibliography}
