\title{Comparison of Finite Volume Method and Particle Tracking Method on the Solution of Complex Solute Transport Problem}
\tocauthor{M. Hokr} \author{} \institute{}
\maketitle
\begin{center}
{\large \underline{Milan Hokr}}\\
Technical University of Liberec\\
{\tt milan.hokr@tul.cz}
\\ \vspace{4mm}{\large Jiri Havlicek}\\
Technical University of Liberec\\
{\tt jiri.havlicek@tul.cz}

\end{center}

\section*{Abstract}

The background of problem we solve is the role of mechanical stress effects on hydraulic properties
of fractured rock \cite{bag}. We consider several variations of 2D fracture network (set of lines in a plane), differing
by aperture and the hydraulic conductivity of fractures over several orders of magnitude, given by set of
different ratios of external mechanical stress field applied in x and y direction \cite{hud}.
On such tasks we solve equations of water flow and solute transport using Flow123D software, developed at
Technical university of Liberec \cite{flow123}. The flow equation is solved by the mixed-hybrid finite element method.
The advective transport equation is solved by the finite volume method, with upwind weighting and
explicit time steps. We evaluate the breakthrough curve and the residence time, that is determined as a
weighted average of time for each part of the mass on the output for each time step of calculation.
We compare our results with calculations produced by the software NAPSAC, using standard finite
element method for flow and  particle tracking method for solute transport calculation. The
residence time value is a direct result of calculation for each particle \cite{pol}.
The present comparison follows the previous work comparing the results of flow only \cite{hokr}. We obtain
a comparison of two principles of the transport solution -- continuous and discrete approach -- to the data
and geometrically complex task. Both methods produce consistent results with diference in ``tails'' of
the breakthrough curves, where only small number of particles and small mass remains in the domain.

\bibliographystyle{plain}
\begin{thebibliography}{10}

\bibitem{bag}
{\sc A. Baghbanan and L. Jing}. {Stress Effects on Permeability in Fractured Rock Mass with Correlated Fracture Length and Aperture}. Int. J. of Rock Mech. and Min. Sci. 45(8), pp. 1320-1334, 2008.



\bibitem{hud}
{\sc J. Hudson and L. Jing and I. Neretnieks}. {Technical Defnition of the 2-D BMT Problem for Task C}. Technical report, DECOVALEX-2011 project, 2008.



\bibitem{flow123}
{\sc O. Severyn and M. Hokr and J. Kralovcova and J. Kopal and M. Tauchman}. {Flow123D: Numerical Simulation Software for Flow and Solute Transport Problems in Combination of Fracture Network and Continuum}. Technical report, Technical University of Liberec, Liberec, Czech Republic, 2008.



\bibitem{pol}
{\sc M. Polak}. {Transport Simulation Using Particle Tracking Method}. Technical report, Progeo Ltd., Czech Republic, 2010.



\bibitem{hokr}
{\sc M. Hokr and J. Kopal and J. Brezina and P. Ralek}. {Sensitivity of Results of the Water Flow Problem in a Discrete Fracture Network with Large Coefficient Differences}. In: I. Dimov et al (Eds.): NMA 2010, LNCS 6046, pp. 420-427, Springer, 2011.

\end{thebibliography}
