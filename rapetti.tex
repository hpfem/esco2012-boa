\title{About Quasi-Static Approximations of Maxwell's Equations}
\tocauthor{F. Rapetti} \author{} \institute{}
\maketitle
\begin{center}
{\large Francesca Rapetti}\\
University of Nice Sophia-Antipolis\\
{\tt francesca.rapetti@unice.fr}
\end{center}

\section*{Abstract}

Maxwell's equations are fundamental for the description
of electromagnetic phenomena and valid over a wide range 
of spatial and temporal scales. The static limit
of the theory is well defined and much easier.
The electric and magnetic fields are given by the laws
of Coulomb and Biot-Savart.
As soon as there is any time dependence, we should in principle
use the full set of Maxwell's equations with all their complexity.
However, a broad range of important applications are described by
some particular models, as the ones in the low frequency range, emerging from neglecting
particular couplings of electric and magnetic field related quantities. 
These applications include motors, sensors, power generators, transformers and
micromechanical systems.
%in order to achieve a preliminar accurate electromagnetic analysis of these devices,
%suitable numerical approximations of Maxwell equations are carried out. 
Note also that the quasi-static models are useful for a better 
understanding of both low frequency electrodynamics and 
the transition from statics to electrodynamics.
We thus present a wider frame to treat 
the quasi-static (QS) limit of Maxwell equations. Following \cite{LL,Melcher,GM}, 
we discuss the fact
that there exists not one but indeed two dual Galilean limits (called ``electric'' or EQS, and ``magnetic'' or MQS limits).   
These limits have however to be suitably coupled to model sofisticated devices such as 
wireless power transfer systems using resonant magnetic coupling \cite{Ka}. We thus propose
to analyse a quasi-static approach   for these devices that are regarded as one of the most promising methods for mid-range wireless charging systems.





\begin{thebibliography}{99}




\bibitem{LL}
M.~Le Bellac, J.-M.~L\'evy-Leblond, 
Galilean Electromagnetism, 
\emph{Il Nuovo Cimento}, \textbf{14} (1973), 217--233.

\bibitem{Melcher} 
J.~R.~Melcher, H.~A.~Haus, \emph{Electromagnetic Fields and Energy},
Prentice Hall (1980).


\bibitem{GM}
M.~de Montigny, G.~Rousseaux, 
On some applications of Galilean electrodynamics of moving bodies,
\emph{Am. J. Phys.}, \textbf{75} (2007), 984--992.


\bibitem{Ka}
R.~E.~Hamam, A.~Karalis, J.D.~Joannopoulos, M.~Solja\v ci\'c,
{\em Efficient weakly-radiative wireless energy transfer: An EIT-like
    approach}, Ann. Phys., {\bf 324} (2009), 1783--1795.


\end{thebibliography}




