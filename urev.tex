\title{Maxwell Equations with New Dissipative Memory Boundary Conditions  }
\tocauthor{M. Urev} \author{} \institute{}
\maketitle
\begin{center}
{\large Mikhail Urev}\\
 Institute of Computational Mathematics and Mathematical Geophysics SB RAS\\
{\tt mih.urev2010@yandex.ru}

\end{center}

\section*{Abstract}


The paper considers the initial-boundary value problem for Maxwell
equations in a bounded domain and finite time interval.
Dissipative memory boundary conditions that are discussed in the
paper are different from those used at present [1], [2]. This
boundary condition was obtained in author's paper [3]

$$
\mathbf{E}_\tau(\mathbf{x},t) = \frac{1}{2\pi}
\sqrt{\frac{\mu}{\sigma} } \frac{\partial}{\partial t}\int_{t_0}^t
\frac{\mathbf{H} (\mathbf{x},\xi) \times \mathbf{n}(\mathbf{x}) d
\xi }{\sqrt{t-\xi} }  \equiv C_\sigma (D^{1/2}_{0+,t} \mathbf{H})
(\mathbf{x},t) \times \mathbf{n}(\mathbf {x}),
$$

where $ D^{1/2}_{0+,t} $ denotes the operator of fractional
differentiation with respect to $t$ of Riemann-Liouville order
$1/2$. In case of a harmonic time dependency of the
electromagnetic field this boundary condition becomes equal to the
classical Leontovich impedance condition. Maxwell operator with
these boundary conditions has been studied in suitable function
spaces and theorem of existence and uniqueness of the
corresponding initial-boundary value problem has been proven.


\bibliographystyle{plain}
\begin{thebibliography}{10}

\bibitem{1urev}
{\sc  M. Fabrizio and A. Morro }. {A boundary condition with memory in electromagnetism}. Arch. Rational Mech. Anal. 136, 359-381 (1996).



\bibitem{2urev}
{\sc S. Nicaise and C. Pignotti}. { Energy decay rates for solutions of Maxwell's system with a memory boundary condition}. Collectanea Mathematica, North America, 58, 1-16 (2007).



\bibitem{3urev}
{\sc  M.V. Urev }. {Boundary conditions for Maxwell equations with arbitrary time dependence}. Comput. Math. Math. Phys., 37:12, 1444-1451 (1997).

\end{thebibliography}
