\title{BEM Solution of MHD Flow in a Pipe Coupled with Magnetic Induction of Exterior Region}
\tocauthor{M. Tezer-Sezgin} \author{} \institute{}
\maketitle
\begin{center}
{\large M{\"u}nevver  Tezer-Sezgin}\\
Middle East Technical University\\
{\tt munt@metu.edu.tr}
\\ \vspace{4mm}{\large Sel\c{c}uk   Han Ayd{\i}n}\\
Karadeniz Technical University\\
{\tt shaydin@ktu.edu.tr}

\end{center}

\section*{Abstract}

The magnetohydrodynamic (MHD) flow through pipes has important applications in electromagnetic pumps, MHD generators, accelerators, flowmeters and blood flow measurements. The fluid flowing through a pipe of cross section in 2D, is viscous, incompressible and electrically conducting under an externally applied magnetic field which is perpendicular to the axis of the pipe. The velocity and the induced magnetic field inside the pipe are influenced with the magnetic induction of outside electrically conducting medium through conducting boundary. The governing equations are obtained with the interaction of Navier-Stokes equations and Maxwell's equations through Ohm's law in the pipe whereas in the outside region induced magnetic field is harmonic. These there coupled equations with the coupled boundary conditions are solved by using boundary element method (BEM) both in and outside of the pipe. Equations for inside of the pipe are transformed to homogeneous modified Helmholtz equations and solved together with the Laplace equation of external region. BEM is suitable especially for infinite regions in which only the interior boundary has to be discretized. This enables one to obtain small sized system for infinite region problems. The solution is obtained for several values of Hartmann number $M$ and magnetic Reynolds numbers ${R_m}_1$ and ${R_m}_2$ for interior and exterior region, respectively. As $M$ increases the flattenning tendency for both the velocity and inside induced magnetic field are observed, and the joining of the two induced magnetic fields on the conducting boundary is achieved with changing values of ${R_m}_1$ and ${R_m}_2$.

\bibliographystyle{plain}
\begin{thebibliography}{10}

\bibitem{lungu33}
{\sc A. Carabineanu and E. Lungu}. {Pseudospectral method for MHD pipe flow}. Inter. Jour. for Num. Meth. in Engg., 68, (2006), 173-191.



\bibitem{munevver33}
{\sc M. Tezer-Sezgin and C. Bozkaya}. {The boundary element solution of magnetohydrodynamic flow in an infinite region}. Jour. CAM, 225, (2009), 510-521.

\end{thebibliography}
