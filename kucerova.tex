\title{Comparison of Optimal Designs of Experiments \\ Suitable for Inverse Analysis}
\tocauthor{A. Kucerova} \author{} \institute{}
\maketitle
\begin{center}
{\large Eli\v{s}ka Janouchov\'a}\\
CTU in Prague, Faculty of Civil Engineering\\
{\tt eliska.janouchova@fsv.cvut.cz}
\\ \vspace{4mm}{\large Tom\'a\v{s} Mare\v{s}}\\
CTU in Prague, Faculty of Civil Engineering\\
{\tt tomas.mares.1@fsv.cvut.cz}
\\ \vspace{4mm}{\large Anna Ku\v{c}erov\'a}\\
CTU in Prague, Faculty of Civil Engineering\\
{\tt anicka@cml.fsv.cvut.cz}

\end{center}

\section*{Abstract}

Design of experiments has a wide usage in the field of inverse
analysis. One possible application appears in sampling-based
sensitivity analysis~\cite{Helton:2006:RESS2}, the other in a
surrogate modelling~\cite{Simpson:2001:EC}. In these situations, two
principal requirements are placed on design of experiments quality:
space-filling properties and orthogonality. To define an optimal
design of experiments, a number of different criteria were formulated
in literature. Some of them prefer mainly the space-filling designs,
e.g.  Audze-Eglais potential energy~\cite{Audze:1977} or Euclidean
maximin distance~\cite{Johnson:1990:JSPI}, others put more emphasis on
minimising the non-orthogonality such as correlation coefficients.

The aim of this contribution is to compare different criteria for an
optimal design formulation. We focus on the quality of the resulting
optimal designs regarding their application in sampling-based
sensitivity analysis and in an artificial neural network training
process. We also compare the computational time required for a
particular optimal design generation taking into account possible
parallelization of a relating optimization process. Finally, we 
discuss the suitability of particular criteria for a design extension
by new sample points.


\bibliographystyle{plain}
\begin{thebibliography}{10}

\bibitem{Helton:2006:RESS2}
{\sc J.C. Helton and J.D. Johnson and C.J. Sallaberry and C.B. Storlie}. {Survey of sampling-based methods for   uncertainty and sensitivity analysis}. Reliab Eng Syst Safe,   91:10-11 (2006), 1175-1209.



\bibitem{Simpson:2001:EC}
{\sc  T.W. Simpson and J. D. Peplinski and P. N. Koch and J. K. Allen}. {Metamodels for computer-based engineering design:   Survey and recommendations}. Engineering with Computers, 17 (2001) 129-150.



\bibitem{Audze:1977}
{\sc P. Audze and V. Eglais}. {New approach for planning   out of experiments}. Problems of Dynamics and Strengths, 35 (1977) 104-107, Zinatne Publishing House.



\bibitem{Johnson:1990:JSPI}
{\sc  M. Johnson and L. Moore and D. Ylvisaker}. {Minimax and maximin distance designs}. J Stat Plan Infer, 26:2 (1990) 131-148.

\end{thebibliography}
