\title{Thermoluminescence Analysis and Estimation of the Kinetics Parameters by the Weibull Distribution in a Diamond-Like Carbon Film}
\author{} \tocauthor{J. Morales} \institute{}
\maketitle
\begin{center}
{\large \underline{Javier Morales}}\\
Universidad Autonoma de Nuevo Leon\\
{\tt tequilaydiamante@yahoo.com.mx}
\\ \vspace{4mm}{\large Roberto Soto}\\
Universidad Autonoma de Nuevo Leon\\
{\tt robsotov@gmail.com}
\\ \vspace{4mm}{\large Javier Almaguer}\\
Universidad Autonoma de Nuevo Leon\\
{\tt almagerjavier@gmail.com}

\end{center}

\section*{Abstract}


Diamond films exhibits excellent properties as high thermal conductivity and low electrical conductivity, due to phonon phenomena.  Also, its strong valence bonds allow gap values up to $5.4 \;ev$. When diamond films are exposed to beta radiation, some electrons pass from the valence to the conduction band, and other stayed trapped between both bands. If the diamond is heated after beta radiation, the trap electrons could be encountered with their respectively holes or positrons emitting light at different frequencies, this phenomenon is called Thermoluminescence TL.   
 The goal in this work is the estimation of the kinetics parameters of the tramp or electrons trapping such as frequency and energy supposing the intensity are a dose and temperature function following a Weibull distribution.

\bibliographystyle{plain}
\begin{thebibliography}{10}

\bibitem{Moralesss}
{\sc J. Morales and R. Bernal and C. Cruz-Vazquez and E. G. Salcido-Romero and V. M. Casta\~no }. {THERMOLUMINESCENCE OF TEQUILA-BASED NANODIAMOND}. Radiation Protection Dosimetry (2010), 1-4.



\bibitem{Furettass}
{\sc C. Furetta and G. Kitis and C.-H. Kuo}. {Kinetics parameters of CVD diamond by computerised glow-curve deconvolution (CGCD)}. Nuclear Instruments and Methods in Physics Research B 160 (2000) 65-72.

\end{thebibliography}
