\title{Using $hp$-adaptive Methods in Shielding Effectiveness Calculations}
\tocauthor{P. Karban} \author{} \institute{}
\maketitle
\begin{center}
{\large Zdenek Kubik}\\
University of West Bohemia\\
{\tt zdekubik@kae.zcu.cz}
\\ \vspace{4mm}{\large David Panek}\\
University of West Bohemia\\
{\tt panek50@kte.zcu.cz}
\\ \vspace{4mm}{\large \underline{Pavel Karban}}\\
University of West Bohemia\\
{\tt karban@kte.zcu.cz}
\\ \vspace{4mm}{\large Jiri Skala}\\
University of West Bohemia\\
{\tt skalaj@kae.zcu.cz}

\end{center}

\section*{Abstract}

This paper deals with the numerical solution of the effectiveness of an electromagnetic shielding~\cite{emc} with the help of the finite element method. Two main objectives are followed. The first one is to propose a new approach to shielding effectiveness  theory based on the high frequency numerical model. Our model uses integral characteristics on contrary to standard models~\cite{celozzi}. The second main goal is to find an appropriate method for the field model calculation.     

The mathematical model consists of partial differential equations describing the distribution of the electromagnetic wave~\cite{hermes-electromagnetics}. Its numerical solution is performed by a fully adaptive higher-order finite element method~\cite{hermes-project} developed by the hpfem.org group. The results are compared with the results obtained by using a commercial software. 

The proposed method is illustrated on the solution of shielding enclosures and the results are verified by experimental measurements.

\bibliographystyle{plain}
\begin{thebibliography}{10}

\bibitem{emc}
{\sc EMC: Electromagnetic Theory to Practical Design}. {P. A. Chatterton, M. A. Houlden}. John Wiley \& Sons, 1992.



\bibitem{celozzi}
{\sc New Figures of Merit for the Characterization of the Performence of Shielding Enclosures}. {S. Celozzi}. IEEE Transactions on Electromagnetic Compatibility, Vol. 46, No. 1., February 2004.



\bibitem{hermes-project}
{\sc P. Solin et al.}. {Hermes - Higher-Order Modular Finite Element System}. http://hpfem.org/.



\bibitem{hermes-electromagnetics}
{\sc L. Dubcova and P. Solin and J. Cerveny and P. Kus}. {Space and Time Adaptive Two-Mesh hp-FEM for Transient Microwave Heating Problems}. Electromagnetics, Vol. 30, p. 23 - 40, 2010.

\end{thebibliography}
