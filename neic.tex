\title{Parallelzation of a PCG-AMG Solver in Multi-CPU/GPU Environments}
\tocauthor{A. Neic} \author{} \institute{}
\maketitle
\begin{center}
{\large \underline{Aurel Neic}}\\
University of Graz\\
{\tt aurel.neic@uni-graz.at}
\\ \vspace{4mm}{\large Manfred Liebmann}\\
University of Graz\\
{\tt manfred.liebmann@uni-graz.at}
\\ \vspace{4mm}{\large Gernot Plank}\\
Medical University of Graz\\
{\tt gernot.plank@medunigraz.at }
\\ \vspace{4mm}{\large Gundolf Haase}\\
University of Graz\\
{\tt gundolf.haase@uni-graz.at}

\end{center}

\section*{Abstract}

We present a parallel conjugate gradient solver with an algebraic multigrid preconditioner for second-order elliptic PDEs called: Parallel Toolbox (PT). The PT is designed for multi-CPU and multi-GPU environments and uses non-overlapping domain decomposition as parallelization approach \cite{ptamg}. 
We discuss in detail the parallelization strategy, the resulting parallel performance properties, and possible future optimizations.
Strong scalability benchmarks are presented for the bidomain equations of a state-of-the-art model of rabbit ventricles using the CARP (Cardiac Arrythmias Research Package) simulator.

\bibliographystyle{plain}
\begin{thebibliography}{10}

\bibitem{ptamg}
{\sc G. Haase and M. Liebmann and C. C. Douglas and and G. Plank }. {A parallel algebraic multigrid solver on graphics processing units}. HPCA (China), Revised Selected Papers, ser. Lecture Notes in Computer Science, W. Zhang, Z. Chen, C. C. Douglas, and W. Tong, Eds., vol. 5938. Springer, 2009, pp. 38-47.

\end{thebibliography}
