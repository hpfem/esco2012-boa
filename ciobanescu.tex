\title{Adaptive Models of Micro-Turbine for Distributed Generation Integration}
\author{} \tocauthor{I. Ciobanescu Husanu} \institute{}
\maketitle
\begin{center}
{\large \underline{Irina Ciobanescu Husanu}}\\
Drexel University, Philadelphia PA, USA\\
{\tt Irina.Ciobanescu@drexel.edu}
\\ \vspace{4mm}{\large Radian Belu}\\
Drexel University, Philadelphia PA, USA\\
{\tt radian.belu@drexel.edu}

\end{center}

\section*{Abstract}

Micro-turbine generation systems are currently attracting a lot of attention as energy sources needed to meet users demands in the distributed generation market mainly due to the deregulation of electric power utilities, advancement in technology, and environmental concerns (1). This paper documents the development and validation of predictive algorithms for modeling the micro-turbine in a building combined cooling, heating and power (BCHP) system. The objective of this project is the development of a simulation model of a micro-turbine generator as a standalone power generation system to validate a micro-turbine generator as part of a hybrid power generation system. In this paper, detailed models of micro-turbine unit (MTU) have been developed under different operating conditions. The dynamic behavior of a micro-turbine is nonlinear, so its operation will become unstable and may even suddenly stop operating if no proper control is adopted (2). Hence, this paper builds a dynamic model that considers parameter uncertainty and nonlinear behavior for micro-turbines. The mathematical models developed here are based on typical up to 30kW natural gas-fired micro-turbine, although, this model could be extended to encompass a micro-turbine of any capacity and fuel-type. In developing the mathematical model we considered the thermodynamic equations that describe the polytropic processes in the compressor and turbine, and the heat and mechanical energy balance equations. A linear analysis method was used to derive the equations that relate the change in the micro-turbine exhaust backpressure to the change in its output power and efficiency.(1) (2) The mathematical model was applied to the baseline performance data found in the literature on the low power micro-turbine unit under steady-state conditions at various loads. Under these modes of operation, the basic operating parameters (temperatures, pressures, voltages, currents, etc.) and the output power of the micro-turbine were measured, and its energy efficiency was calculated (1) (2). In this paper we focused more on the functionality and accuracy of the system model rather than on the functionality of the component models. 

\bibliographystyle{plain}
\begin{thebibliography}{10}

\bibitem{Grillo-Massucco-Morini-1}
{\sc Grillo S. and Massucco S. and Morini A. and Pitto A. and Silvestro F.}. {Microturbine Control Modeling to Investigate the Effects of Distributed Generation in Electric Energy Networks}. IEEE Systems Journal, 2010, Vol. 4, (3), pp. 303-312.



\bibitem{Wanik-Erlich-2}
{\sc Wanik M.Z.C. and Erlich I.}. {Dynamic Simulation of Microturbine Distributed Generators integrated with Multi-Machines Power System Network}. 2nd IEEE International Conference on Power and Energy 2008 Malaysia.

\end{thebibliography}
