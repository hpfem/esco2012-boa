\title{Modeling of Moisture Transfer in Soils}
\tocauthor{T. Krejci} \author{} \institute{}
\maketitle
\begin{center}
{\large \underline{Tomas Krejci}}\\
Faculty of Civil Engineering, Czech Technical University in Prague\\
{\tt krejci@cml.fsv.cvut.cz}
\\ \vspace{4mm}{\large Tomas Koudelka}\\
Faculty of Civil Engineering, Czech Technical University in Prague\\
{\tt koudelka@cml.fsv.cvut.cz}

\end{center}

\section*{Abstract}

A numerical model describing coupled hydro-mechanical behaviour of soils is presented. The paper
deals with micro-mechanics-based model based on Lewis and Schrefler's approach of heat and mois-
ture transfer in deforming porous media [1]. The theory of deformation of soils (soil skeleton) and
other porous materials comes out from the concept of effective stresses [2]. Soils consists generally
from three components - grains (skeleton), liquid (water) and gas (water vapour and air). Total stress
in soil can be expressed as a combination of effective stress between grains and pore water pressure
and pore gas pressure.
Three types of equations must be solved during non-linear coupled transport problems. There
are constitutive equations (retention curves, material properties), transport equations (Fick's law and
Darcy's law) and continuity equations. After discretization of driving equations using finite elements
method, a system of non-symmetric and non-linear algebraic equations is obtained, even when defor-
mation of the solid is linear elastic. Iteration is then required within each time step.
The moisture transfer is a very slow process. Therefore, in case of linear consolidation no iteration
is necessary within time steps. There exists experimental evidence that such a description of consoli-
dation by the linear elastic model for a porous medium along with the constant permeability could not
be realistic. At least the permeability is mostly subject of variation reflecting the dependence of the
permeability on void ratio. The CAM clay model with the bilinear form of the normal consolidation
line (NCL) is then adopted as a suitable model which describes the effect of over-consolidation and
structure strength on time dependent processes in soils.
The numerical model is applied to a computer simulation of a soil consolidation process. The
wide area of application can be found, e.g, in the structure - subsoil interaction influenced by changes
of ground water level.


\bibliographystyle{plain}
\begin{thebibliography}{10}

\bibitem{[1]}
{\sc R. W.  Lewis and B. A. Schrefler.}. {The finite element method in static and dynamic deformation and consolidation of porous media.}. John Wiley \& Sons, Chiester-Toronto (1998)..



\bibitem{[2]}
{\sc J. \v{S}ejnoha et. al.}. {Structure - subsoil interaction in view of transport processes in porous media.}. CTU Reports 1(5) (2001)..

\end{thebibliography}
