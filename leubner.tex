\title{Computer Modeling of Electromagnetic Acceleration of Ferromagnetic Bodies in $hp$-FEM software Hermes}
\author{} \tocauthor{K. Leubner} \institute{}
\maketitle
\begin{center}
{\large \underline{Karel Leubner}}\\
Czech Technical University, Prague, Czech Republic\\
{\tt leubnkar@fel.cvut.cz}
\\ \vspace{4mm}{\large Franti\v sek Mach}\\
West Bohemia University, Pilsen, Czech Republic\\
{\tt fmach@kte.zcu.cz}

\end{center}

\section*{Abstract}

The paper deals with the mathematical and computer modeling of electromagnetic acceleration of ferromagnetic bodies. This principle of acceleration is known for more than $150$ years and is used in a lot of various industrial and other applications. The basic applications are electromagnetic accelerators and electromagnetic actuators. \cite{eafb}

The model was created with the help of $hp$-FEM software Hermes \cite{hermesxx} and a lot of own procedures and scripts. The models of both devices are similar. They consist of a system of two strongly nonlinear ordinary differential equations describing the time evolution of the current in the field circuit and motion of the accelerated body, and two nonstationary partial differential equations describing the distribution of electromagnetic and temperature fields in the system. The model is then solved numerically and the methodology is illustrated by a typical example whose results are discussed.

\bibliographystyle{plain}
\begin{thebibliography}{10}

\bibitem{eafb}
{\sc K. Leubner}. {Electromagnetic acceleration of ferromagnetic bodies}. Acta Technica CSAV, vol. 54, no. 4, p. 343-357.



\bibitem{hermesxx}
{\sc P. Solin et al.}. {Hermes - Higher-Order Modular Finite Element System}. http://hpfem.org.

\end{thebibliography}
