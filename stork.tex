\title{State-Space Energy Based Structure of Generalized Brayton-Moser Nonlinear Network Representations}
\author{} \tocauthor{M. Stork} \institute{}
\maketitle
\begin{center}
{\large Daniel Mayer, Josef Hru\v{s}\'{a}k, \underline{Milan \v{S}tork}}\\
University of West Bohemia\\
{\tt stork@kae.zcu.cz}

\end{center}

\section*{Abstract}

It is well known that the dynamical behavior of a linear or nonlinear electrical network may in principle be described by a set of state equations together with the so called output equation. The output equation determines the amount of information about the internal state contained in the observed output signal. Any such mathematical structure is called state space representation of the network. The state variables may be taken as coordinates of a state vector x(t) referred to a basis in a Euclidean space of n dimensions, called the state space X of the system representation R{S}. Each solution of the state equations corresponding to an initial state and/or to an input control signal determines a state space trajectory \cite{kalman}. 

From a physical point of view to any such trajectory corresponds some course of a total energy accumulated in the actual state of the given system. Any state space representation of a system having a particularly simple or useful structure is called canonical. One of the most useful canonical state space representations in electrical network theory are known as Brayton-Moser equations \cite{stork-hrusak-mayer}. It is well known that a relatively large class of real world systems, at least in principle, admits to start the modeling phase of the given situation using Hamilton`s principle of clasical mechanics either in the form of Euler-Lagrange equations or in the form of Hamilton canonical equations. In spite of the fact that the Hamilton principle has originally been formulated for class of mechanical systems only, it is now well known that from a general point of view the specific physical nature of the system does not play any essential role. On the other hand it is important to notice that in the process of deriving both the sets of equations a crucial role plays a proper representation of the energy storage in the state space of the system.In this paper, a class of linear and especially nonlinear systems is discussed. A conceptually new approach is based on the idea that abstract state space energy can be measured by a distance of the actual state $x(t)$ from the equilibrium. The proposed state space energy based approach seems to open a new perspective in development of sufficiently universal and more adequate abstract system representations for variety of natural systems. Crucial feature of the proposed approach is that all the derived results have to be structurally consistent with two fundamental laws of nature -- the causality principle as well as with the energy conservation principle.
%[1]     R. E. Kalman, “Mathematical description of linear dynamical systems”, SIAM Journal of Control, 1, pp.152-192, 1963.[2]     R. K. Brayton, and J. K. Moser, “A theory of nonlinear networks - 1”, Quarterly of Applied Mathematics, Vol.22, No.1, pp. 1-33, 1964. [3]     R. K. Brayton, and J. K. Moser, “A theory of nonlinear networks - 2”, Quarterly of Applied Mathematics, Vol.22, No.2, pp. 81-104, 1964. [4]     A. G. J. MacFarlane, “Dynamical system models”, George G. Harrap \& Co. Ltd., London, Toronto, pp. 363-382, Great Britain, 1970.[5]     D. Mayer, “The state variable method of electrical network analysis“ ACTA TECHNICA CSAV, No.6, pp.761-789, 1970.[6]  P. Tabuada, G. J. Pappas, "Abstractions of Hamiltonian control systems“, Automatica of IFAC, 39, pp. 2025-2033, 2003.[7]  D. Jeltsema, J. M. A. Scherpen, "A dual relation between port-Hamiltonian systems and the Brayton-Moser equations for nonlinear switched RLC circuits“, Automatica of IFAC, 39, pp. 969-979, 2003.[8]     J. Hrusak, “Anwendung  der  Äquivalenz  bei Stabilitätsprüfung“, Tagung  ü. die Regelungstheorie, Mathematisches Forschungsinstitut, Oberwolfach, Universitaet  Freiburg,  West. Germany, 1969.[9]     J. Hrusak, “The isometric transformations method and some of its applications“,  PhD Thesis, CTU Prague, pp.1-137, ( In Czech ), 1971.[10]     J. Hrusak, M. Stork, D. Mayer, “Generalized Tellegen's Principle and state space energy based causal systems description”, in Advances in Energy Research: Distributed Generations Systems Integrating Renewable Energy Resources, Part I, Basic theory and advanced approaches, Chapter 4., pp. 95-139, NOVA Science Publ., USA, 2011.

\bibliographystyle{plain}
\begin{thebibliography}{10}

\bibitem{kalman} R. E. Kalman: Mathematical description of linear dynamical systems. SIAM Journal of Control, 1, pp.152-192, 1963.
\bibitem{stork-hrusak-mayer}
{\sc M. Stork, J. Hrusak and D. Mayer}: {Generalized Tellegen's Principle and state space energy based causal systems description}. Advances in Energy Research: Distributed Generations Systems Integrating Renewable Energy Resources, Part I, Basic theory and advanced approaches, Chapter 4., pp. 95-139, NOVA Science Publ., USA, 2011..

\end{thebibliography}
