\title{An Optimal Eigenvalue Solver}
\tocauthor{J. Gedicke} \author{} \institute{}
\maketitle
\begin{center}
{\large Joscha Gedicke}\\
Dipl.-Math.\\
{\tt gedicke@math.hu-berlin.de}

\end{center}

\section*{Abstract}

This talk presents the optimal computational complexity of a combined adaptive finite
element method (AFEM) with an iterative algebraic eigenvalue solver for a simple
symmetric model problem. The analysis is based on a direct approach for eigenvalue
problems and allows the use of higher order conforming finite element spaces with fixed
polynomial degree. First the quasi-optimal convergence for the eigenvalue problem
under the usual assumption that the sub-problems are solved exactly is shown. As for
the source problem the convergence analysis for the quasi error does not need the inner
node property. These results are relaxed to the inexact approximations of some iterative
eigenvalue solver and thus lead to a combined AFEM and iterative eigenvalue solver
algorithm. The proposed optimal algorithm involves a proper termination criterion for
the iterative algebraic eigenvalue solver and does not need any coarsening. Numerical
examples show optimal computational complexity.
This contribution is joint work with Carsten Carstensen (HU Berlin, Germany).

\bibliographystyle{plain}
\begin{thebibliography}{10}

\bibitem{CG_OptimalAFEMES}
{\sc C. Carstensen and J. Gedicke}. {An Adaptive Finite Element Eigenvalue Solver of Quasi-Optimal Computational Complexity}. MATHEON Preprint 662 (2009), TU Berlin.

\end{thebibliography}
