\title{The Finite Cell Method for Transient, Non-linear Heat Conduction}
\tocauthor{N. Zander} \author{} \institute{}
\maketitle
\begin{center}

{\large Nils Zander, Stefan Kollmannsberger, Ernst Rank}\\
Technische Universit\"{a}t M\"{u}nchen\\
{\tt \{zander, kollmannsberger, rank\}@bv.tum.de}

\\

{\large Patrick Erbts, Alexander D\"{u}ster}\\
Technische Universit\"{a}t Hamburg-Harburg\\
{\tt \{patrick.erbts, alexander.duester\}@tu-harburg.de}

\\



\end{center}

\section*{Abstract}

Since the early years of computational engineering, the classical Finite
Element Method (FEM) has become the state-of-the-art approach to solve initial
boundary value problems numerically. 
Although major enhancements allowed for highly sophisticated
simulations, the idea to geometrically
resolve the physical domain on the discretization level remained unchanged.
In cases of complex geometries, this intrinsic need for a conform mesh
is still a burden in today's engineering practice.


The recently introduced Finite Cell Method (FCM) overcomes this problem by
combining the benefits of high-order Finite Elements ($p$-FEM) with the idea of
fictitious domains \cite{Parvizian:07.1}.
The approach embeds  the possibly complex physical domain $\Omega_{phys}$
in a fictitious domain $\Omega_{fict}$ and solves the problem on their simply
shaped union. 
The localization factor $\alpha$ allows to recover the original geometry on the 
integration level. 
Recent research results confirm the excellent applicability
of this new method in the fields of
non-linear continuum mechanics, 
topology optimization, 
thin-walled structures, 
bone mechanics
and advection-diffusion problems.
Also the method's potential for multi-physical problems has been demonstrated
in the context of linear thermoelasticity.

In the present work, the Finite Cell Method is employed to solve the
transient, non-linear heat equation on non-conforming meshes.
In particular, the weak enforcement of Dirichlet boundary conditions is
addressed.
The results of a conventional $p$-FEM simulation serve as reference and the
convergence characteristics of both approaches are compared.

\bibliographystyle{plain}
\begin{thebibliography}{10}

\bibitem{Parvizian:07.1}
{\sc J. Parvizian and A. D\"uster and and E. Rank}. {Finite Cell Method-h- and p-extension for embedded domain problems in solid mechanics.}. Computational Mechanics, 41:121-133, 2007..

\end{thebibliography}
