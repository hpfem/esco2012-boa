\title{Algebraic Multigrid in Fluid-Structure Interaction and Convection Dominated Problems}
\author{} \tocauthor{M. Gee} \institute{}
\maketitle
\begin{center}
{\large Michael Gee}\\
Technische Universit\"at M\"unchen\\
{\tt gee@tum.de}

\end{center}

\section*{Abstract}

In the last decades algebraic multigrid principles have been extensively
applied in various real world single field applications. Less attention though has been paid to adopting AMG principles to the specific needs of multifield problems. In this talk, I
focus on two recent developments made in the field of algebraic
multigrid (AMG) methods:

The first is a new monolithic AMG scheme for the implicit solution
of fluid-structure interaction (FSI) simulations
\cite{geekuettlerwall}. Therein, an AMG hierarchy for the
nonsymmetric monolithic fluid, structure and mesh movement system of
equations is constructed that also considers a coarse representation
of interfield off-diagonal coupling blocks in a variationally
consistent way. This  FSI multigrid preconditioner is based on a
mixed smoothed aggregation and
nonsymmetric smoothed aggregation approach \cite{SalaTuminaro,wiesnertuminarogee} that
accounts for the hybrid nature of the monolithic FSI problem. It
limits the amount of artificial smearing of individual fields on
coarse levels by construction and allows for field specific
smoothers within its block oriented smoothing procedure. It is
mainly utilized in the very challenging regime of soft tissue
biomechanics of the vascular and respiratory system where
disadvantageous density ratios and large deformations appear.

The second recent development is a framework for constructtion of algebraic transfer operator for highly non symmetric systems that arise from the discretization of convection dominated problems such as e.g. the (Navier-) Stokes equations \cite{wiesnertuminarogee}.
This framework follows a Schur complement perspective as this is suitable for 
both symmetric and nonsymmetric systems.  In particular, a connection between 
algebraic multigrid and approximate block factorizations is explored. This 
connection demonstrates that the convergence rate of a two-level model 
multigrid iteration is completely governed by how well the coarse 
discretization approximates a Schur complement operator. The new grid 
transfer algorithm is then based on computing a Schur complement but
restricting the solution space of the corresponding grid transfers
in a Galerkin-style so that a far less
expensive approximation
is obtained. The final
algorithm corresponds to a Richardson-type iteration that is used to 
improve a simple initial prolongator or a simple initial restrictor. 

\bibliographystyle{plain}
\begin{thebibliography}{10}

\bibitem{geekuettlerwall}
{\sc M.W. Gee and U. K\"uttler and W.A. Wall}. {Truly Monolithic Algebraic Multigrid for Fluid-Structure Interation}. International Journal for Numerical Methods in Engineering, (2010), 85, 987-1016.



\bibitem{wiesnertuminarogee}
{\sc T. Wiesner and R.S. Tuminaro and W.A. Wall and M.W. Gee }. {Multigrid Transfers for Nonsymmetric Systems based on Schur Complements and Galerkin Projections}. (2011), Numerical Linear Algebra With Applications, submitted.



\bibitem{SalaTuminaro}
{\sc M.~Sala and R.S. Tuminaro}. {A new petrov-galerkin smoothed aggregation preconditioner for   nonsymmetric linear systems}. Siam J. Scientific Comp, (2008), 31, 143--166.

\end{thebibliography}
