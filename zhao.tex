\title{Numerical Methods for Studying Magnetic Flux Compression Generators}
\tocauthor{Q. Zhao} \author{} \institute{}
\maketitle
\begin{center}
{\large \underline{Qiang Zhao}}\\
Institute of Applied Physics and Computational Mathematics, Beijing 100094\\
{\tt Q.ZHAO.CN@gmail.com}
\\ \vspace{4mm}{\large Zhiwei Dong}\\
Institute of Applied Physics and Computational Mathematics, Beijing 100094\\
{\tt dong\_zhiwei@iapcm.ac.cn}

\end{center}

\section*{Abstract}

Magnetic flux compression generators (MFCGs) power supplies have been used for a variety of applications\cite{ref001}. A typical generator is constructed by a metallic armature and an outer helical coil. The armature filled with high explosive expands outward, compressing magnetic flux initially established between it and the outer helical coil. The inductance decrease in the helical winding is a result both of the decrease in radial separation between the armature and the coil and of the decrease in coil axial length as the armature short circuits successive turns of the coil. This decrease in inductance, combined with the coil's tendency to conserve flux, leads to amplification of the electrical current. Chemical energy stored in the explosive is converted to kinetic energy of the armature. This energy in turn is transformed by the coil magnetic field into electrical energy delivered to the external load\cite{ref002,ref003,ref004}. 

The report describes the construction of APMFCG, a code designed to study the characteristics of helically wound MFCGs, in which the axial field generated by a solenoidal conductor system. The code combines an Eulerian materials response code, which can treat elastic-plastic flows, high explosives, energy flows, and many other effects of importance to the MFCG problem,  with a two-dimensional magnetic field solver to compute the self-consistent interaction between the field and the conductors, including magnetic forces, Joule heating, nonlinear resistive diffusion, and the effects of the external load. The solver employs variable zoning designed to provide resolution with a minimum number of meshes, and incorporates flux-continuity transport to reduce spurious numerical diffusion\cite{ref005,ref006}. Representative calculations illustrate some of the feature of APMFCG, and include both computer-generated plots and numerical listings.

\bibliographystyle{plain}
\begin{thebibliography}{10}

\bibitem{ref001}
{\sc H. Knoepfel}. {Pulsed High magnetic Fields}. North-Holland Publishing Company, Amsterdam, 1970.



\bibitem{ref002}
{\sc C. M. Fowler}. {Megagauss Physics}. Science, Vol. 180, no. 4083, pp. 261-267 (1973).



\bibitem{ref003}
{\sc J. W. Shearer and et al.}. {explosive driven magnetic field compression generators}. J.Appl. Phys., Vol. 39, no. 4, pp. 2102-2116 (1968).



\bibitem{ref004}
{\sc J.C.Crawford and R.A.Damerow}. {explosively driven high energy generators}. J. Appl. Phys., Vol. 39, no. 11, pp.5224-5231 (1968).



\bibitem{ref005}
{\sc Qiang Zhao and Guangwei Yuan}. {Analysis and construction of cel-centered finite volume scheme for diffusion equations on distored meshes}. Computer Methods in Applied Mechanics and Engineering, 198 (2009) 3039-3050.



\bibitem{ref006}
{\sc Qiang Zhao and et al.}. {A cell-centered diffusion finite volume scheme and its application to magnetic flux compression generator}. World Academy of Science, Engineering and Technology, issue 71, 2010, pp.327-332.

\end{thebibliography}
