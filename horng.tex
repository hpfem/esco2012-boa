\title{An Error Minimized Pseudospectral Penalty Direct Poisson Solver}
\tocauthor{T.-L. Horng} \author{} \institute{}
\maketitle
\begin{center}
{\large Tzyy-Leng Horng}\\
Feng Chia University\\
{\tt tlhorng123@gmail.com}

\end{center}

\section*{Abstract}

This paper presents a direct Poisson solver based on an error minimized Chebyshev pseudospectral
penalty formulation for problems defined on rectangular domains. The current method is based
on eigenvalue-eigenvector matrix diagonalization methods developed long time before and elaborated
by Chen, Su, and Shizgal (2000). However, this kind of fast Poisson solvers utilizing pseudospectral
discretization is usually restricted to periodic or Dirichlet boundary conditions only for
efficient implementation. Enforcement of Neumann or Robin boundary conditions requests messive
row operations that reduces the easiness of operation. The current method can easily accommodate all
three types of boundary conditions: Dirichlet, Neumann, and Robin boundary conditions. The reason
for that is that it employs penalty method. Besides, the penalty parameters are determined analytically
such that the discrete L2 error is minimized. 2D and 3D numerical experiments are conducted and the
results show that the penalty scheme computes numerical solutions with better accuracy, compared
to the traditional approaches with boundary conditions enforced strongly. The current method can
be extended to generally linear 2nd order elliptic-type partial differential boundary value problems
with arbitrary coefficients with only one restrction that those arbitrary coefficients must be able to be
separable to coordinate-variable functions. Under this generalization, this method can be surely applied
to solve famous Helmholtz equation too. As to computing efficiency, the asymptotic operation
count for current method in 3D case is 2NxNyNz(Nx + Ny + Nz), and its counterpart for 2D case
is 2NxNy(Nx + Ny). Obviously, it is far superior to method expressing Laplace operator in tensor
product. For an FFT-based method the asymptotic operation count is basically 2NxNyNz(log(Nx) +
log(Ny) + log(Nz)) for 3D case, and 2NxNy(log(Nx) + log(Ny)) for 2D case. Indeed, the current
method which relies on extensive matrix-matrix multiplications is inferior to FFT-based methods
in theory. However, this inferiority also depends on hardware. For moderate Nx, Ny, and Nz, it is not
necessarily inferior in computers nowadays. Of course, for large grid resolution, FFTbased methods
remain the the best choice if handling boundary conditions is not a problem.

\bibliographystyle{plain}
\begin{thebibliography}{10}

\bibitem{Chen-Su-Shizgal}
{\sc H. Chen and Y. Su and B.D. Shizgal}. {A Direct Spectral Collocation Poisson Solver in Polar and Cylinder Coordinates}. J. Comput. Phys. 160 (2000) 453-469.

\end{thebibliography}
