\title{A Posteriori Error Estimation for Rational Eigenvalue Problems Arising in Photonic Band Structure Calculations}
\tocauthor{C. Engstrom} \author{} \institute{}
\maketitle
\begin{center}
{\large Christian  Engstr\"om}\\
Ume\aa\, University\\
{\tt christian.engstrom@math.umu.se}
\\ \vspace{4mm}{\large Luka Grubi\v{s}i\'c}\\
University of Zagreb\\
{\tt luka.grubisic@math.hr}

\end{center}

\section*{Abstract}

Operator functions with a nonlinear dependence of a spectral parameter, but a linear dependence on the field arise in numerous applications in science. In this talk, we focus on a posteriori error estimation for self-adjoint rational operator functions with periodic coefficients \cite{Eng1}. The main applications are metallic photonic crystals and metamaterials, which are promising materials for controlling electromagnetic waves. Using perturbation theory for generalized Rayleigh quotients and the Cauchy integral analysis of the generalized resolvent we develop a posteriori approximation error estimation techniques and derive a computable residual estimate of a finite element approximation \cite{Luk1}. The effectivity of the estimator is shown and the derived refinement indicator is successfully used in numerical computations of band structures. 

\bibliographystyle{plain}
\begin{thebibliography}{10}

\bibitem{Eng1}
{\sc C. Engstr\"om}. {On the spectrum of a holomorphic operator-valued function with applications to absorptive photonic crystals}. Math. Mod. Meth. Appl. S. 20 (2010) 1319-1341 .



\bibitem{Luk1}
{\sc L. Grubi\v{s}i\'c and J. Ovall}. {On Estimators for Eigenvalue/Eigenvector Approximations}. Math. Comp. 78 (2009) 739-770.

\end{thebibliography}
