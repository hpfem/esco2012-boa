\title{An Optimal Control Approach to Enforcing the Eikonal Equation}
\tocauthor{C. Basting} \author{} \institute{}
\maketitle
\begin{center}
{\large \underline{Christopher Basting}}\\
University Erlangen-Nuremberg\\
{\tt christopher.basting@am.uni-erlangen.de}
\\ \vspace{4mm}{\large Dmitri Kuzmin}\\
University Erlangen-Nuremberg\\
{\tt kuzmin@am.uni-erlangen.de}

\end{center}

\section*{Abstract}

A finite element level set method for problems with evolving interfaces is presented. Special emphasis is laid on preserving the distance function
property of the level set function. In contrast to the standard two-step
approach (convection + reinitialization), an optimal control approach is
proposed. The first method to be discussed imposes the Eikonal equation
as a constraint on the weak solution to the transport equation.
The constrained variational formulation incorporates a nonlinear
(anti-)diffusive term that adjusts the gradients of the level set
function whenever they become too steep or too flat. To ensure the
well-posedness of the discrete problem, the correction term may need
to be deactivated in regions where the gradient of the level set
function changes its direction. In the second method, the strong
form of the Eikonal equation is replaced with a PDE-constrained optimization problem. The cost functional is defined so as to
minimize the residual of the Eikonal equation in the least-squares
sense. The optimality conditions are written as a nonlinear system
of equations for the primal solution, control variable, and the dual
solution. No ad hoc deactivation of selected elements is required in
this version of the constrained level set algorithm. The resultant
nonlinear saddle-point problems are solved in an iterative fashion.
A numerical study is performed for representative test problems. 
\nocite{glowinski-tallec}
\nocite{becker-meidner-vexler}

\bibliographystyle{plain}
\begin{thebibliography}{10}

\bibitem{glowinski-tallec}
{\sc R. Glowinski and Patrick Le Tallec}. {Augmented Lagrangian and operator-splitting methods in nonlinear mechanics}. Society for Industrial Mathematics; ISBN-13: 978-0898712308.

\bibitem{becker-meidner-vexler}
{\sc R. Becker and D. Meidner and B. Vexler}. {Efficient numerical solution of parabolic optimization problems by finite element methods}. Optimization Methods and Software, 22:813-833, 2007

\end{thebibliography}
