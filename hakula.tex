\title{$hp$-Adaptive Eigensolution: Benchmark Studies}
\tocauthor{H. Hakula} \author{} \institute{}
\maketitle
\begin{center}
{\large \underline{Harri Hakula}}\\
Aalto University\\
{\tt Harri.Hakula@aalto.fi}
\\ \vspace{4mm}{\large Tomi Tuominen}\\
Aalto University\\
{\tt tatuomin@math.hut.fi}

\end{center}

\section*{Abstract}

In this paper we discuss the use of $hp$-adaptive methods of Houston \& S\"uli-type 
\cite{HSS} in the context of eigenproblems. 
Using a combination of bubble modes and Sobolev-regularity estimation
we arrive at an adaptive algorithm controlling both $h$ and $p$. 
In broad terms our work can also be interpreted as an extension of \cite{GO} to 
high-order FEM.

For an eigenpair $(\mu,\textbf{u})$ we define a discrete variational problem
(omitting the customary use of $h$): Find $\textbf{u} \in V$ and $\mu \in \mathbb{R}$ such that
$a(\textbf{u},\textbf{v}) = \mu\,m(\textbf{u},\textbf{v}), \forall\, \textbf{v} \in V$, where
$a(\cdot,\cdot)$ and $m(\cdot,\cdot)$ are the bilinear forms of stiffness and mass matrices, respectively.

Let us assume that $V$ corresponds to standard elements of $p=4$.
Then we consider an extension $V^+$ with bubbles for $p=5,6,7$, say, and compute a local (elemental)
error estimator $\boldsymbol{\epsilon}$: 
Find $\boldsymbol{\epsilon} \in V^+$ such that
$a(\textbf{u}+\boldsymbol{\epsilon},\textbf{v}) = \mu\,m(\textbf{u},\textbf{v}), \forall\, \textbf{v} \in V^+$,
i.e., $a(\boldsymbol{\epsilon},\textbf{v})~=~\mu\,m(\textbf{u},\textbf{v}) - a(\textbf{u},\textbf{v})$.

In this work we present results on two benchmark problems: isospectral drums (Laplace operator) and
rotor blade (piece of a cylindrical shell). These results are then compared with the best
a priori results in both cases.


\bibliographystyle{plain}
\begin{thebibliography}{10}

\bibitem{HSS}
{\sc P. {Houston} and B. {Senior} and E. {S\"uli}}. {Sobolev regularity estimation for $hp$-adaptive finite element methods}. Proceedings of ENUMATH 2001, pp. 631--656..



\bibitem{GO}
{\sc L. Grubisic and J. S. Ovall.}. {On estimators for eigenvalue/eigenvector approximations}. {Math. Comp.}, 78 (2009), 739--770.

\end{thebibliography}
