\title{Modeling of Loud-Speaker with $hp$-Adaptive Methods}
\tocauthor{L. Koudela} \author{} \institute{}
\maketitle
\begin{center}
{\large Lukas Koudela}\\
University of West Bohemia\\
{\tt koudela@kte.zcu.cz}
\\ \vspace{4mm}{\large Pavel Karban}\\
University of West Bohemia\\
{\tt karban@kte.zcu.cz}
\\ \vspace{4mm}{\large Antonin Predota}\\
University of West Bohemia\\
{\tt apredota@kte.zcu.cz}
\\ \vspace{4mm}{\large Oldrich Turecek}\\
University of West Bohemia\\
{\tt turecek@ket.zcu.cz}
\\ \vspace{4mm}{\large Ladislav Zuzjak}\\
University of West Bohemia\\
{\tt zuzjak@ket.zcu.cz}

\end{center}

\section*{Abstract}

This paper deals with the numerical solution of the acoustic field. Two objectives are followed. The first one is to create a computational model of the loud-speaker with respecting the possibility of diaphragm stiffness. \cite{1kk} The second goal is to obtain directional characteristics of the model with the help of the finite element method. \cite{3kk}

The mathematical model is described by the Helmhotz differential equation describing
the distribution of the acoustic field. The numerical solution is performed by a fully adaptive higher-order finite element method developed by the hpfem.org group. \cite{2kk} The results of modeling are compared with the results obtained using a commercial software and verified by measurement.

The measurement is realized in the anechoic chamber and based on standard. \cite{4kk} The low-frequency speaker is used in suitable frequency range to avoid the diaphragm uncontrolled breakup. 

\bibliographystyle{plain}
\begin{thebibliography}{10}

\bibitem{1kk}
{\sc J. Borwick}. {Loudspeaker and Headphone Handbook}. Focal Press 2001.



\bibitem{2kk}
{\sc P.Solin et al.}. {Hermes - Higher-Order Modular Finite Element System}. http://hpfem.org/.



\bibitem{3kk}
{\sc P. Solin and J. Cerveny and I. Dolezel}. {Arbitrary-Level Hanging Nodes and Automatic Adaptivity in the hp-FEM}. Math. Comput. Simul., Vol. 77, p. 117-132, 2008.



\bibitem{4kk}
{\sc CSN JC 60268-5}. {Electroacoustic devices}. Part 5: Loudspeakers.

\end{thebibliography}
