\title{Coupled Modeling of Geothermal Heat Production in Fault Zones}
\tocauthor{E. Holzbecher} \author{} \institute{}
\maketitle
\begin{center}
{\large Ekkehard Holzbecher}\\
Georg August Universit\"at G\"ottingen\\
{\tt eholzbe@gwdg.de}

\end{center}

\section*{Abstract}

Geothermal heat production from deep reservoirs (5000-7000 m) is currently examined within the collaborative research program "Geothermal Energy and High-Performance Drilling” (gebo), funded by the Ministry of Science and Culture of Lower Saxony (Germany) and Baker Hughes. The projects concern exploration and characterization of geothermal reservoirs as well as production. They are gathered in the four major topic fields: geosystem, drilling, materials, technical system. We present modelling of a generic model set-up concerning the fluid flow heat flux through and within a fault zone. 
For modelling we use the commercial COMSOL Multiphysics code. The code allows the coupling of various flow and transport modes. Mainly we link the Darcy-mode for porous media flow with heat transfer in porous media, where heat convection and heat conduction are taken into account. The numerical calculations of COMSOL Multiphysics are based on the method of Finite Elements, for which various different options are explored by changing numerical parameters.
We present a comparison between numerical and analytical solutions concerning fluid flow in a system consisting of a porous matrix and a single fracture. A sensitivity study provides a quantitative criterion, under which conditions fluid flux through the system is fracture dominated or matrix dominated. The streamfunction approach is utilized to obtain streamlines through the entire system of matrix and fracture. 
Moreover, a doublet is examined as the usual installation for geothermal heat production. In deep reservoirs, doublet flow patterns arise in natural faults, in artificially produced fracs or in deep aquifers. In extension of well-known characteristics of such situations we examine coupling effects in our numerical models.
The paper concludes with a discussion of coupling effects in fault zones concerning geothermal heat production. 


\bibliographystyle{plain}
\begin{thebibliography}{10}

\bibitem{Groundwater Lowering}
{\sc E. Holzbecher and Y. Jin}. {Borehole Pump \& Inject}. Int. J. Environmental Protection.

\end{thebibliography}
