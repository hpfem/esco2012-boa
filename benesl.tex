\title{Numerical Modeling of the Flow Structures around Thin Body in  the  Stratified Fluid}
\tocauthor{L. Benes} \author{} \institute{}
\maketitle
\begin{center}
{\large Lud\v{e}k Bene\v{s}}\\
Dept. of Technical Mathematics CTU Prague, Karlovo   n\'{a}m. 13, CZ-121\,35 Prague 2\\
{\tt benes@marian.fsik.cvut.cz}
\\ \vspace{4mm}{\large Tom\'{a}\v{s} Bodn\'{a}r}\\
Dept. of Technical Mathematics CTU Prague, Karlovo   n\'{a}m. 13, CZ-121\,35 Prague 2, Czech Republic\\
{\tt bodnar@marian.fsik.cvut.cz}
\\ \vspace{4mm}{\large Philippe Frauni\'{e}}\\
Universit\'{e} du Sud Toulon-Var, Laboratoire de Sondages Electromagn\'{e}tiques de l'Environnement Terrestre, B\^{a}timent F, BP 132, 83957 La Garde Cedex, France\\
{\tt Philippe.Fraunie@lseet.univ-tln.fr}

\end{center}

\section*{Abstract}

 Numerical simulation of an internal gravity waves past a moving body in a
 stably stratified flow is performed in   comparison with  laboratory
  experiments. The flow field in the towing tank  with a moving thin
  horizontal strip is modeled using different computational ways and
   different numerical schemes. 

The  mathematical model is based on the 
Boussinesq approximation of the averaged Navier--Stokes equations.  The
resulting set of partial 
differential equations is then solved  by two different numerical schemes. The
first method is the second-order finite volume AUSM MUSCL scheme
combined with the artificial compressibility method in dual time . For the
time integration the second order BDF method is used in physical time, while
the third order Runge-Kutta method is used in artificial time. The second
scheme is based on the high order compact finite-difference discretizations.
The time integration is carried out by the Strong Stability Preserving
Runge-Kutta scheme. 

The developing of the internal waves is studied by different ways. The
obstacle is supposed either as the stationary in the incoming flow or as the
moving body in the stratified fluid.
 In the simplest case, the thin body is modeled only by the appropriate boundary
 condition, then via penalization technique or by simple immersed boundary
 method.  All computations are compared  each other and also with the experiment. 

\bibliographystyle{plain}
\begin{thebibliography}{10}

\bibitem{ben1}
{\sc  L. Bene\v{s} and J. F\''{u}rst and Ph. Frauni\'{e}}. {Comparison of Two  Numerical Methods for the Stratified Flow}. Computers \& Fluids Vol. 46, Issue 1,  2011, p.148-154.



\bibitem{ben2}
{\sc T. Bodn\'{a}r and L. Bene\v{s} and Ph. Frauni\'{e} and K. Kozel}. {Application of   Compact Finite--Difference Schemes to Simulations of Stably Stratified Fluid Flows}. Applied Mathematics and Computations, doi:10.1016/j.amc.2011.08.058, in press.

\end{thebibliography}
