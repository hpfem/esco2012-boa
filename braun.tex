\title{Comparing Different Approaches to the Numerical Solution of the Poisson Equation in 3D via the Method of Finite Elements}
\tocauthor{M. Braun} \author{} \institute{}
\maketitle
\begin{center}
{\large Moritz Braun}\\
University of South Africa\\
{\tt moritz.braun@gmail.com}

\end{center}

\section*{Abstract}

The accurate solution of the three dimensional Poisson equation 
$
-\nabla^2 \Phi = 4 \pi \rho
$
is important in many  areas of science and engineering:
\begin{enumerate}
\item Eletrical Engineering: Design of complex semiconductor devices
\item Molecular and Theoretical Solid State Physics: Input for density
functional approach
\item 
Chemistry of Solutions  and Molecular Dynamics: Ingredient for solving the Poisson-Boltzmann equation 
\end{enumerate}
Efficient solution methods are needed since self-consistent calculations
require the repeated solution of the Poisson equation.\\
In this contribution we consider the weak formulation of the Poisson equation
\cite{SolinBook} 
and compare the solutions 
obtained using  different polynomial orders $p$ of the FEM
as well as for different types of grid and for both 
direct and iterative solvers for the resulting linear equation. We also
investigate the performance of different iterative algorithms as
provided by pysparse\cite{pysparse}.
We finally  report on results obtained with  the factorization approach for the case of a cartesian tensor grid 
pioneered by Berger and Sundholm \cite{NonIterFact} adjusted to non-zero boundary conditions, that promises to allow a direct solution even for very large linear systems.

\bibliographystyle{plain}
\begin{thebibliography}{10}

\bibitem{SolinBook}
{\sc P. Solin}. {Partial Differential Equations and the Finite Element Method}. John Wiley and Sons, 2004.



\bibitem{pysparse}
{\sc R. Geus and D. Wheeler and D. Orban}. {Pysparse Documentation Release 1.0.2}. http://pysparse.sf.netPysparse.pdf.



\bibitem{NonIterFact}
{\sc R. J. F. Berger and D. Sundholm }. {A non-iterative Numerical Solver of Poisson and Helmholtz Equations in Three Dimensions}. Advances in Quantum Chemistry, {\bf 50} 234-246 (2005).

\end{thebibliography}
