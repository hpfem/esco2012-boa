\title{Some Computational Aspects of Smooth Approximation}
\tocauthor{K. Segeth} \author{} \institute{}
\maketitle
\begin{center}
{\large Karel Segeth}\\
Technical University of Liberec\\
{\tt segeth@math.cas.cz}

\end{center}

\section*{Abstract}

The contribution is devoted to the problem of smooth approximation of data in 1D. We are concerned with the exact interpolation of the data at nodes and, at the same time, with the smoothness of the interpolating curve and its derivatives.

The interpolating curve is defined as the solution of a variational problem with constraints. The problem of approximation of data does not have a unique solution as our requirements on the smoothness of the approximating curve can be very subjective. The smooth approximation of this kind can lead, e.g., to the cubic spline interpolation.

We discuss the proper choice of basis systems for this way of approximation and present the results of several 1D numerical examples that show the advantages and drawbacks of smooth approximation.

\bibliographystyle{plain}
\begin{thebibliography}{10}

\bibitem{tg}
{\sc A. Talmi and G. Gilat}. {Method for Smooth Approximation of Data}. J. Comput. Phys. 23 (1977) 93-123.

\end{thebibliography}
