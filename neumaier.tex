\title{Worst Case Error Bounds for the Solution of Uncertain Poissson 
Equations with Mixed Boundary Conditions}
\tocauthor{A. Neumaier} \author{} \institute{}
\maketitle
\begin{center}
{\large \underline{Arnold Neumaier}}\\
Faculty of Mathematics, University of Vienna, Austria\\
{\tt Arnold.Neumaier@univie.ac.at}
\\ \vspace{4mm}{\large Tanveer Iqbal}\\
Faculty of Mathematics, University of Vienna, Austria\\
{\tt Tanveer.Iqbal@univie.ac.at}

\end{center}

\section*{Abstract}

We consider the solution of linear elliptic partial differential 
equations with mixed boundary conditions on 2-dimensional domains with 
a polygonal boundary, not necessarily convex. The equation may contain 
uncertain parameters constrained by inequalities.

We show how to use finite element approximations to compute worst case 
a posteriori error bounds for linear response functionals determined 
by the solution. All discretization errors are taken into account. 

Our bounds are based on the dual weighted residual (DWR) method of 
{\sc Becker and Rannacher} \cite{BecR}, and treat the uncertainties 
with the optimization approach described in {\sc Neumaier} \cite{Neu}. 
To get the error bounds, we use a first order formulation whose 
solution with linear finite elements produces compatible piecewise 
linear approximations of the solution and its gradient. For each 
iteration of the optimization procedure, we need to solve three related
boundary value problems, from which we produce the bounds. No knowledge 
of domain-dependent apriori constants is necessary.

We implemented the method for Poisson-type equations with an uncertain 
mass distribution and mixed Dirichlet/Neumann boundary conditions.


\bibliographystyle{plain}
\begin{thebibliography}{10}

\bibitem{BecR}
{\sc R. Becker and R. Rannacher}. {An optimal control approach to a posteriori error estimation in finite
element methods}. pp. 1--102 in:
Acta Numerica 2001 (A. Iserles, ed.), Cambridge Univ. Press 2001..



\bibitem{Neu}
{\sc A. Neumaier}. {Certified error bounds for uncertain elliptic equations}. J. Comput. Appl. Math. 218 (2008), 125--136.

\end{thebibliography}
