\title{Automatic Tracking of Corona Propagation in Three-dimensional Simulations of Non-normal Drop Impact on a Liquid Film}
\author{} \tocauthor{P. Brambilla} \institute{}
\maketitle
\begin{center}
{\large \underline{Paola Brambilla}}\\
Politecnico di Milano\\
{\tt paola.brambilla@mail.polimi.it}
\\ \vspace{4mm}{\large Giulio Romanelli}\\
Politecnico di Milano\\
{\tt romanelli@aero.polimi.it}
\\ \vspace{4mm}{\large Alberto Guardone}\\
Politecnico di Milano\\
{\tt alberto.guardone@polimi.it}

\end{center}

\section*{Abstract}

The splashing of a liquid drop into a liquid film results in a variety of fluid dynamics phenomena, whose understanding is far from being complete, due to both the complexity of the splashing phenomenon itself and to the large number of flow parameters influencing it. These are the e.g. Weber, Ohnesorge and Reynolds number, which are computed from the drop velocity, density, superficial tension, dynamic viscosity and size \cite{YAR2006}.
To investigate a large number of flow parameters, relevant flow features are to be captured automatically during the simulations and used to characterized the flow regime.  In normal impacts, namely, in the case of drop velocity normal to the liquid film and parallel to the gravity field, the height and radius of the corona propagating away from the center of impact is often chosen to describe the splashing regime \cite{NIK2007}.
In present paper, a new criteria for describing in a simple way the dynamics of the splashing is proposed for oblique drop impacts, with the drop velocity angle with respect to the liquid film equal to 20, 40, 60 and 80 degrees.

\bibliographystyle{plain}
\begin{thebibliography}{10}

\bibitem{NIK2007}
{\sc N. Nikolopoulus et al}. {Three-dimensional numerical investigation of droplet impinging normally onto a wall film}. J.Comput.Phys. 225 (2007) 322-341.



\bibitem{YAR2006}
{\sc A. L. Yarin}. {Drop impact dynamics: splashing, spreading, receding, bouncing...}. Ann.Rev.Fluid.Mech 38 (2006) 159-192.

\end{thebibliography}
