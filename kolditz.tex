\title{OpenGeoSys: An Open Source Initiative for Numerical Simulation of Thermo-Hydro-Mechanical/Chemical (THM/C) Processes in Porous Media}
\tocauthor{O. Kolditz} \author{} \institute{}
\maketitle
\begin{center}
{\large Olaf Kolditz}\\
Helmholtz Centre for Environmental Research UFZ / TU Dresden\\
{\tt olaf.kolditz@ufz.de}

\end{center}

\section*{Abstract}

In this paper we describe the OpenGeoSys (OGS) project, which is a scientific open source initiative for numerical simulation of thermo-hydro-mechanical-chemical (THMC) processes in porous media. The basic concept is to provide a flexible numerical framework (using primarily the Finite Element Method (FEM)) for solving multi-field problems in porous and fractured media for applications in geoscience and hydrology. To this purpose OGS is based on an object-oriented FEM concept including a broad spectrum of interfaces for pre- and post-processing. The OGS idea has been in development since the mid eighties. We provide a short historical note about the continuous process of concept and software development having evolved through Fortran, C, and C++ implementations. The idea behind OGS is to provide an open platform to the community, outfitted with professional software engineering tools such as platform-independent compiling and automated benchmarking. A comprehensive benchmarking book has been prepared for publication. Benchmarking has been proven to be a valuable tool for cooperation between different developer teams, e.g. for code comparison and validation purposes (DEVOVALEX and CO2BENCH projects). On one hand, object-orientation (OO) provides a suitable framework for distributed code development; however the parallelization of OO codes still lacks efficiency. High-performance-computing (HPC) efficiency of OO codes is subject to future research.

\bibliographystyle{plain}
\begin{thebibliography}{10}

\bibitem{kolditz-o}
{\sc KOLDITZ O et al.}. {OpenGeoSys: An open source initiative for numerical simulation of thermo-hydro-mechanical/chemical (THM/C) processes in porous media}. Environ. Earth Sci., CLEAN Thematic Issue.

\end{thebibliography}
