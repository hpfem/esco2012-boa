\title{Bioelectromagnetic Simulation for Everyone}
\tocauthor{J. Starzynski} \author{} \institute{}
\maketitle
\begin{center}
{\large \underline{Jacek Starzynski}}\\
Warsaw University of Technology\\
{\tt jstar@iem.pw.edu.pl}
\\ \vspace{4mm}{\large Robert Szmuro}\\
Warsaw University of Technology\\
{\tt szmurlor@iem.pw.edu.pl}
\\ \vspace{4mm}{\large Bartosz Chaber}\\
Warsaw University of Technology\\
{\tt chaberb@iem.pw.edu.pl}

\end{center}

\section*{Abstract}

Cloud computation technologies open a new perspective for scientific computing. Sophisticated software
can now be made available as on-demand service, reducing costs and broadening accessibility for end-users.
The paper presents an implementation of a system serving open source electromagnetic field simulation software
as a web based service.
The primary users of the system are medical staff and electromagnetic
safety engineers which are usually not familiar with field simulation methods. The user friendliness of a system
combined with its flexibility, scalability and extendibility is realized with help of carefully designed
concept of configurable usage-scenarios. Our aim is to present this scenarios concept in detail.


\bibliographystyle{plain}
\begin{thebibliography}{10}

\bibitem{Nierhoerster2009}
{\sc Oliver Niehörster et. al.}. {Providing Scientific Software as a Service in Consideration of Service Level Agreements}. Proceedings of the Cracow Grid Workshop (CGW), pp. 55-63, 2009.

\end{thebibliography}
