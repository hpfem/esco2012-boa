\title{Fractional Dynamics and Anomalous Diffusion, Case Studies Simulations}
\author{} \tocauthor{O. Fernandez} \institute{}
\maketitle
\begin{center}
{\large \underline{Oscar  Fernandez}}\\
Universidad Autonoma de Nuevo Leon\\
{\tt omega\_dos\_veces@hotmail.com}
\\ \vspace{4mm}{\large Javier Almaguer}\\
Universidad Autonoma de Nuevo Leon\\
{\tt almagerjavier@gmail.com}
\\ \vspace{4mm}{\large Roberto Soto}\\
Universidad Autonoma de Nuevo Leon\\
{\tt robsotov@gmail.com}
\\ \vspace{4mm}{\large Hector  Flores}\\
Universidad Autonoma de Nuevo Leon\\
{\tt serolfrotceh@googlemail.com}

\end{center}

\section*{Abstract}


In this work we present 
simulations and numerical
results related with some 
systems which follow an 
anomalous diffusion dynamics. 
In the mathematical model for 
the classical diffusion the 
size of jump and the time 
elapsed between successive jumps 
are random variables that 
follow -typically- the normal 
distribution (for the size of the jumps) 
and the exponential distribution 
(for the time between jumps). 
On the other hand, in the model 
of anomalous diffusion the size 
of the jumps and or the time 
between successive jumps follow 
a distribution wich asymptotically 
resemble a power law. Such is the 
case of the Levy stable distribution. 
In such cases, the dynamics of the 
diffusive processes will be governed 
by a fractional partial differential 
equation, with the  index of the 
power law distribution being
the order of fractional derivative.

\bibliographystyle{plain}
\begin{thebibliography}{10}

\bibitem{Metzler}
{\sc Ralf Metzler and Joseph Klafter}. {The random walk's guide to anomalous diffusion: a fractional dynamics approach.}. Physics Reports  339 (2000) 1-77.

\end{thebibliography}
