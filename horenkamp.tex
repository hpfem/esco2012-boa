\title{Efficient Approximation of Coherent Pairs}
\author{} \tocauthor{C. Horenkamp} \institute{}
\maketitle
\begin{center}
{\large \underline{Christian Horenkamp}}\\
University of Paderborn\\
{\tt horenc@math.upb.de}
\\ \vspace{4mm}{\large Michael Dellnitz}\\
University of Paderborn\\
{\tt dellnitz@math.upb.de}

\end{center}

\section*{Abstract}

Transport phenomena in non-autonomous dynamical systems appear in a variety of applications e.g. the transport of water mass induced by eddies in the oceanic fluid flow.In order to treat transport phenomena of non-autonomous dynamical systems so-called transfer operator methods have been developed in the last years. For instance a method for the detection of so-called coherent pairs, which represent slowly mixing structures, has been introduced by Froyland et al. \cite{FroSanMon10}. These transfer operator methods have in common that they involve long term simulations of trajectories on the whole state-space which are computationally expensive. In principle the method of Froyland et al. can be applied to a subpart of the state-space which contains one of the sets of a coherent pair. If the transport process is slow enough we show that transport phenomena over a fixed (long) time horizon imply the existence of almost invariant sets over shorter time intervals. In this talk we use this fact to formulate an algorithm that preselects a part of the state-space as a candidate containing one of the sets of a coherent pair and therefore significantly reduces the related numerical effort.

\bibliographystyle{plain}
\begin{thebibliography}{10}

\bibitem{FroSanMon10}
{\sc G. Froyland and N. Santitissadeekorn and A. Monahan}. {Transport in time-dependent dynamical systems: Finite-time coherent sets}. Chaos, 20(4):043116, 2010.

\end{thebibliography}
