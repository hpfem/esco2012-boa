\title{An Iterative Finite Element Method with Adaptivity for Multiple Eigenvalues}
\tocauthor{S. Giani} \author{} \institute{}
\maketitle
\begin{center}
{\large Stefano Giani}\\
University of Nottingham\\
{\tt Stefano.Giani@nottingham.ac.uk}
\\ \vspace{4mm}{\large Pavel Solin}\\
University of Nevada\\
{\tt solin@unr.edu}

\end{center}

\section*{Abstract}

We consider the task of resolving accurately the $n$th eigenpair of a generalized
eigenproblem rooted in some elliptic partial differential equation (PDE), using an adaptive finite
element method (FEM). 
Conventional adaptive FEM algorithms call a generalized eigensolver after
each mesh refinement step. This is not practical in our situation since the generalized eigensolver
needs to calculate $n$ eigenpairs after each mesh refinement step, it can switch the order of eigenpairs,
and for repeated eigenvalues it can return an arbitrary linear combination of eigenfunctions from
the corresponding eigenspace. Especially when adaptivity is used to target an eigenpair of a multiple
eigenvalue, the change in the linear combination of eigenfunctions may reduce the convergence rate of 
the method.

In order to circumvent these problems, we propose a novel adaptive
algorithm that only calls the eigensolver once at the beginning of the computation, and then employs
an iterative method to pursue a selected eigenvalue-eigenfunction pair on a sequence of locally refined
meshes. Both Picard's and Newton's variants of the iterative method are presented. The underlying
partial differential equation (PDE) is discretized with higher-order finite elements (hp-FEM) but
the algorithm also works for standard low-order FEM. 


\bibliographystyle{plain}
\begin{thebibliography}{10}

\bibitem{luka}
{\sc L. Grubisic And J. S. Ovall}. {On estimators for eigenvalue/eigenvector approximations}. Mathematics of Computation, 78:266 (2008), pp. 739-770.



\bibitem{hp}
{\sc P. Solin and D. Andrs and J. Cerveny And M. Simko}. {PDE-independent adaptive hp-fem based on hierarchic extension of finite element spaces}. J. Comput. Appl. Math., 233 (2010), pp. 3086-3094..



\bibitem{solin}
{\sc P. Solin and K. Segeth and I. Dolezel}. {Higher-order finite element methods}. Chapman \& Hall, CRC Press, London, 2003..

\end{thebibliography}
