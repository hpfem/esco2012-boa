\title{Numerical Simulation of Generalized Oldroyd-B and Generalized Newtonian Flows}
\tocauthor{V. Prokop} \author{} \institute{}
\maketitle
\begin{center}
{\large Vladimir  Prokop}\\
Faculty of Mechanical Engineering, CTU in Prague\\
{\tt prouki@seznam.cz}
\\ \vspace{4mm}{\large Karel Kozel}\\
Faculty of Mechanical Engineering, CTU in Prague\\
{\tt karel.kozel@fs.cvut.cz}

\end{center}

\section*{Abstract}

This work concerns with numerical simulation of generalized Newtonian and generalized Oldroyd-B fluids.
Newtonian model of a fluid is not able to capture all the
phenomena in many fluids with complex microstructure, where are involved much
larger scales than the atomic scale.  Polymeric fluids are the most studied class of
non-Newtonian fluids.
They are characterized by the presence of long chain molecules
stretched out by the drag forces of the surrounding fluid. The retraction of
molecules from stretched state induces an elastic force contributing to the
stress tensor. In polymeric fluids exposed to shearing, polymeric molecules are aligned with
the flow direction causing normal stress in the flow direction \cite{renardy}.

The motion of polymeric fluids is described by the conservation of mass and
momentum. One shall assume that the fluid is incompressible and temperature
variations are negligible. The mathematical model is
complete when a constitutive law, relating extra stress tensor $\bf{T}$ to the motion, is
prescribed. 
When one considers viscoelastic behavior of polymeric fluids, the extra
stress tensor depend not only on the current motion of the fluid, but also on
the history of the motion \cite{bodnar}. In this case the extra stress tensor
${\bf T}$ is decomposed into its Newtonian part ${\bf T}_s$ and its elastic part ${\bf
  T}_e$. Components of the elastic part of the extra stress tensor are computed using
the Oldroyd-B constitutive equation \cite{owens}:

The resulting systems of equations for pressure, velocity components and
components of the elastic part of the extra stress tensor are solved using the
artificial compressibility method. For discretization of spatial
derivatives a finite volume method is employed \cite{prokop}. Inviscid fluxes are discretized in central manner,
viscous fluxes are computed using dual finite volume cells of diamond type. The resulting system
of ODEs is solved using three-stage Runge-Kutta method \cite{prokop} with steady boundary conditions.

\bibliographystyle{plain}
\begin{thebibliography}{10}

\bibitem{renardy}
{\sc M. Renardy}. {Mathematical analysis of viscoelastic flows}. SIAM, 2000.



\bibitem{owens}
{\sc R. G. Owens and T. N. Phillips}. {Computational Rheology}. Imperial College Press, 2002.



\bibitem{prokop}
{\sc V. Prokop and K. Kozel}. {Numerical Simulation of Generalized Newtonian Flows in Bypass}. ICNAAM 2010, AIP Conference Proceedings, Volume 1281, pp 127-130, CD-ROM ISBN: 978-0-7354-0831-9.



\bibitem{bodnar}
{\sc T. Bodnar and A. Sequeira and M. Prosi}. {On the shear-thinning and viscoelastic effects of blood flow under various flow rates}. Applied Mathematics and Computation 217, pp. 5055--5067, 2011.

\end{thebibliography}
