\title{The Behavior of Nanorods Suspended in 3D Flowing Liquid}
\tocauthor{A. Hijazi} \author{} \institute{}
\maketitle
\begin{center}
{\large Abbas Hijazi}\\
Lebanese Universty, Faculty of Sciences I, Beirut, Lebanon\\
{\tt abhijaz@ul.edu.lb}
\\ \vspace{4mm}{\large Saleem Hamady}\\
Lebanese University, Faculty of Sciences I, Beirut, Lebanon\\
{\tt saleem.hamady@gmail.com}
\\ \vspace{4mm}{\large Ali Atwi}\\
Lebanese University, Faculty Of Sciences I, Beirut, Lebanon\\
{\tt atwiali23@hotmail.com}
\\ \vspace{4mm}{\large Antoine Khater}\\
Univerasite du Maine, Laboratoire LPEC, 72085 Le Mans, France\\
{\tt antoin.khater@univ-lemans.fr}

\end{center}

\section*{Abstract}

The dynamics and mobility of nanorods flowing through pores or near surfaces is a subject of fundamental importance for physical and biological studies. This theme has also promising new applications in the context of nanotechnology, biophysics and medicine ranging from cancer treatment and targeted drug delivery to medical imaging. 
	The aim of this work is to study the behavior of nanorods suspended at dilute concentration in a liquid flowing in a 3-dimensional simple shear flow, near a flat solid surface. The dynamics of these nanorods are effectively determined by three independent forces, the first is the hydrodynamic force stemming from the shear flow of the liquid, the second is of thermal origin that causes Brownian diffusion, and the third is due Van der Waals forces arising from dipole - dipole interaction. Additionally, a repulsive interaction springs up, due to Pauli principle, when the nanorods are just about to touch the surface.
	Simulation techniques ranging from Taylor expansion and random walks to Monte Carlo simulation were employed to extract the probability distribution function of the nanorods as a function of their orientation and elevation.


\bibliographystyle{plain}
\begin{thebibliography}{10}

\bibitem{Atwi-Hijazi-Khater-1}
{\sc A. Atwi and A. Khater and A. Hijazi}. {Three-dimensional Monte Carlo simulations of the dynamics of macromolecular particles in solutions flowing in mesopores}. Central European Journal Of Chemistry, 8 (2010) 1009-1013  .



\bibitem{Hijazi-Khater-2}
{\sc A. Hijazi and A. Khater}. {Boeder PDF Brownian simulations for macromolecular rod-like particles near uneven solid surfaces}. European Polymer Journal , 44 (2008) 3409-3416 .

\end{thebibliography}
