\title{Modelling of TE and TM Modes in Photonic Crystal Wave-Guides}
\tocauthor{K. Schmidt} \author{} \institute{}
\maketitle
\begin{center}
{\large \underline{Kersten Schmidt}}\\
DFG-Research center Matheon, TU Berlin\\
{\tt kersten.schmidt@math.tu-berlin.de}
\\ \vspace{4mm}{\large Dirk Klindworth}\\
DFG-Research center Matheon, TU Berlin\\
{\tt klindworth@math.tu-berlin.de}

\end{center}

\section*{Abstract}

Photonic crystals (PhC) are materials comprising a periodic modulation
of the refractive index in the order of the wavelength of the light. %
PhCs have attracted much attention due to their exceptional ability to
engineer the properties of light propagation. %
Light is guided efficiently in waveguides which are formed by omitting
one or a few rows of holes in an otherwise perfectly PhC of finite
size. %
In this project we are interested to compute the TE and TM modes for a
given frequency $\omega$ in those PhC wave-guides, which is a
quadratic eigenvalue problem in the quasi-momentum~$k$ in a unit cell
which is an infinite strip~\cite{Schmidt.Kappeler:2010}. %
Similar quadratic eigenvalue problems for finite unit cells have been
studied already~\cite{Engstroem.Richter:2009}. %
We study different model reductions concerning the infiniteness of the
unit cell with respect to their algorithmic realisation, their
accuracy and computational effort. %
These include the supercell method, a Dirichlet-to-Neumann operator of
an semi-infinite photonic crystal~\cite{Joly.Li.Fliss:2006} as well as
a specially adapted basis on infinite cells. %
The comparison will be supported by numerical experiments with the
{\em p}-version of the finite element method on meshes with curved
cells.

\bibliographystyle{plain}
\begin{thebibliography}{10}

\bibitem{Joannopoulos:1995}
{\sc J.D. Joannopoulos and St. G. Johnson}. {{Photonic Crystals: Molding the Flow of Light}}. Princeton University Press, 2008.



\bibitem{Schmidt.Kappeler:2010}
{\sc K. Schmidt and R. Kappeler}. {{Efficient Computation of Photonic Crystal Waveguide Modes with Dispersive Material}}. Opt. Express, 18: 7307-7322, March 2010.



\bibitem{Engstroem.Richter:2009}
{\sc C. Engstr\"om and Marcus Richter}. {On the Spectrum of an Operator Pencil with Applications to Wave Propagation in Periodic and Frequency Dependent Materials}. SIAM J. Appl. Math., 70 (1), 231-247, 2009.



\bibitem{Joly.Li.Fliss:2006}
{\sc P. Joly and J.R. Li and S. Fliss}. {Exact boundary conditions for periodic waveguides containing a local perturbation}. Comm. Comp. Phys, 1(6), 945--973, 2006.

\end{thebibliography}
